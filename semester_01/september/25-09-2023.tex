\documentclass{article}
\usepackage[utf8]{inputenc}

\usepackage[T2A]{fontenc}
\usepackage[utf8]{inputenc}
\usepackage[russian]{babel}

\usepackage{tabularx}
\usepackage{amsmath}
\usepackage{pgfplots}
\usepackage{geometry}
\geometry{
    left=1cm,right=1cm,top=2cm,bottom=2cm
}
\newcommand*\diff{\mathop{}\!\mathrm{d}}

\newtheorem{definition}{Определение}
\newtheorem{theorem}{Теорема}

\DeclareMathOperator{\sign}{sign}

\usepackage{hyperref}
\hypersetup{
    colorlinks, citecolor=black, filecolor=black, linkcolor=black, urlcolor=black
}

\title{Теория принятия решений}
\author{Лисид Лаконский}
\date{September 2023}

\begin{document}
\raggedright

\maketitle

\tableofcontents
\pagebreak

\section{Лекция — 25.09.2023}

\subsection{Доминирование стратегий}

Сложность решения матричной игры возрасатает с увеличением размеров матрицы, поэтому перед решением игры следует проанализровать матрицу с целью сокращения ее размерности. При анализе матрицы \textbf{следует выделить стратегии, являющиеся дублирующими и заведомо невыгодными игрокам}.

\begin{definition}
    Стратегия $A_{i}$ игрока А доминирует какую-либо его стратегию $A_{l}$, если $a_{i j} \ge a_{l j}$; $\forall j = \overline{1, n}$.
    Строгое доминирование: $(a_{i j} > a_i j)$
\end{definition}

\begin{definition}
    Стратегия $A_{i}$ игрока А дублирует какую-либо его стратегию $A_{l}$, если $a_{i j} = a_{l j}$; $\forall j = \overline{1, n}$
\end{definition}

Оставлять все дублирующиеся стратегии нельзя, \textbf{нужно оставить только одну из них}. В случае доминирования стратегии \textbf{необходимо оставлять только доминирующую стратегию} $A_{i}$, а доминируемую стратегию необходимо отбросить (то есть, удалить строку из матрицы).

\begin{definition}
    Стратегия $B_{j}$ игрока Б доминирует какую-либо его стратегию $B_{j}$, если $a_{i j} \le a_{i l}$; $\forall i = \overline{1, m}$. Мы исходим из того, что игрок Б должен удалять из матрицы большие, а не меньшие столбцы. Стратегию $B_{j}$ называется доминирующей, а стратегия $B_{l}$ называется доминируемой.
\end{definition}

\subsubsection{Определение доминирования в классе смешанных стратегий}

\begin{definition}
    Стратегия $x'$ игрока $A$ доминирует какую-либо его стратегию $x''$, если $x' p^{j} \ge x'' p^{j}$; $\forall j = \overline{1, n}$, где $p^{j}$ — векторы вероятностей.
\end{definition}

Аналогичное определение можно дать и для игрока Б:

\begin{definition}
    Стратегия $y'$ игрока $B$ доминирует какую-либо его стратегию $y''$, если $P_{i} y' \le P_{i} y''$, $\forall i = \overline{1, m}$, где $P_{i}$ — $i$-ая строка матрицы $P = (a_{i j})_{m \times n}$.
\end{definition}

\subsubsection{Примеры}

\paragraph{Пример №1}

Допустим, $P = \begin{pmatrix}
    5 & 2 & 4 \\
    7 & 3 & 6 \\
    1 & 5 & 3 \\
    7 & 3 & 6
\end{pmatrix}$. Проанализируем данную матрицу на предмет доминирования и дублирования. Видим, что $A_{2} = A_{4}$ — следовательно, одну из строк можем убрать. Сократим четвертую строку: $P' = \begin{pmatrix}
    5 & 2 & 4 \\
    7 & 3 & 6 \\
    1 & 5 & 3 \\
\end{pmatrix}$. Кроме того, видим, что вторая строка в матрице строго доминирует первую строку: $A_{2} > A_{1}$. Значит, необходимо удалить строку №1. Получим матрицу: $P'' = \begin{pmatrix}
    7 & 3 & 6 \\
    1 & 5 & 3 \\
\end{pmatrix}$.

Таким образом, эта игра может быть решена как матрица размерности $(2 \times 3)$, при этом можем записать $x = (0, p, 1 - p, 0)$.

\paragraph{Пример №2}

Допустим, $P = \begin{pmatrix}
    -5 & 3 & 1 & 20 \\
    5 & 5 & 4 & 6 \\
    -4 & -2 & 0 & -5
\end{pmatrix}$. Проанализируем данную матрицу на предмет доминирования и дублирования. $A_{3}$ строго доминируется $A_{2}$, следовательно, ее сокращаем: $P' = \begin{pmatrix}
    -5 & 3 & 1 & 20 \\
    5 & 5 & 4 & 6 \\
\end{pmatrix}$. Больше строки невозможно сократить, но перейдем к анализу столбцов: $B_{1} \le B_{2}$, сокращаем второй столбец. Кроме того, $B_{3} < B_{2}$, $B_{4} < B_{3}$. Получаем итоговую матрицу: $P'' = \begin{pmatrix}
    -5 & 1 \\
    5 & 4 \\
\end{pmatrix}$. Первая строка становится доминируемой, переходим к матрице: $P''' = \begin{pmatrix}
    5 & 4 \\
\end{pmatrix} \implies V_{A} = 4$, $(A_2, B_3)$.

\subsubsection{Теоремы о доминировании стратегий}

\begin{theorem}
    Если в матричной игре стратегия $x'$ одного из игроков доминирует его оптимальную стратегию $x^{*}$, то стратегия $x'$ также является оптимальной.
\end{theorem}

\begin{theorem}
    Если стратегия $x^{*}$ одного из игроков является оптмальной, то она недоминируема строго.

    Обратное неверно: $P = \begin{pmatrix}
        1 & 0 \\
        0 & 2
    \end{pmatrix}$. В этом примере две недоминированные стратегии игроков составляют их оптимальные стратегии.
\end{theorem}

\subsection{Решение матричной игры размерности $n \times n$}

\begin{definition}
    Стратегия $A_{i}$ $(B_{j})$ игроков А или B называется \textbf{существенной} (активной) стратегией, если она входит с ненулевой вероятностью в вектор оптимальных смешанных стратегий: $p_{i}^{*} > 0$ ($q_{j}^{*} > 0$).
    Никакая существенная стратегия не может быть доминируемой. В спектр оптимальной смешанной стратегии любого игрока входят только существенные чистые стратегии.
\end{definition}

\begin{definition}
    \textbf{Стратегия} $x$ (y) игрока $A$ (или B) \textbf{называется вполне смешанной}, если ее спектр состоит из всех стратегий игрока $A$ (или B).
\end{definition}

\begin{definition}
    \textbf{Ситуация равновесия} $(x^{*}, y^{*})$ \textbf{называется вполне смешанной}, если ее стратегии $x^{*}$ и $y^{*}$ вполне смешаны.
\end{definition}

\begin{theorem}
    Вполне смешанная $(m \times n)$ игра \textbf{имеет единственное решение и квадратную матрицу} ($m = n$). То есть, ни одна стратегия не является доминируемой и для всех них вероятности ненулевые. Покажем это: так как у игроков все стратегии существенные, то по свойству №1 оптимальных смешанных стратегий если один из игроков придерживается своей оптимальной стратегии, а другой нет, то выигрыш остается неизменным и равным цене игры.

    Рассмотрим решение для игрока $A$: Пусть игрок $A$ выбирал свою оптимальную смешанную стратегию $x^{*}$, состоящую из чисел $P_{1}^{*}, \dots, P_{n}^{*}$. Игрок $B$ выбирал чистые стратегии. Тогда $v = H_{A}(x^{*}, j)$, $\forall j = \overline{1, n}$. То есть, $v = a_{1 j} P_{1}^{*} + a_{2 j} P_{2}^{*} + \dots + a_{m j} P_{m}^{*}$, $\sum\limits_{i = 1}^{m} P_{i}^{*} = 1$.

    Единственное решение будет только при $m = n$.
\end{theorem}

Допустим, $m = n$: \begin{equation}
    \begin{cases}
        a_{1 1} p_{1} + a_{21} p_{2} + \dots + a_{n 1} p_{n} = v \\
        \dots \\
        a_{1 n} p_{1} + a_{2 n} p_{2} + \dots + a_{n n} p_{n} = v \\
        p_{1} + p_{2} + \dots + p_{n} = 1
    \end{cases}
\end{equation}

Решение данной системы выполняется методом обратной матрицы. Введем следующий вектор, состоящий из единиц: $u = (1, \dots, 1)$, перепишем последнее уравнение системы: $1p_{1} + 1p_{2} + \dots + 1p_{n} = 1 \implies x u^{T} = 1 \implies xP = vu = (v, v, \dots, v)$. Отсюда следует, что $x^{*} = v u P^{-1}$. Остается лишь найти $v$: $x^{*} u^{T} = 1 = v u P^{-1} u^{T} \implies v = \frac{1}{u P^{-1} u^{T}} \implies x^{*} = \frac{u P^{-1}}{u P^{-1} u^{T}}$. Таким образом, чтобы найти вектор оптимальных смешанных стратегий, необходимо обратить матрицу игрока $P$ и выполнить записанные в формуле манипуляции. Это применимо только для квадратных матриц.

Решение для $B$: Игрок $B$ выбирает свою оптимальную стратегию $y^{*} = (q_{1}^{*}, \dots, q_{n}^{*})$, а игрок $A$ выбирает свою чистую стратегию $A_{i}$ ($i = \overline{1, n}$). Таким образом, $v = H_{A} (i, y^{*}) =$  \begin{equation}
    \begin{cases}
        v = a_{11} q_{1}^{*} + \dots + a_{1n} q_{n}^{*} \\
        \dots \\
        v = a_{n1} q_{1}^{*} + \dots + a_{n n} q_{n}^{*} \\
        q_{1}^{*} + q_{2}^{*} + \dots + q_{n}^{*} = 1
    \end{cases}
\end{equation} $\implies u y^{*^{T}} = 1$,  $P y^{*^{T}} = v u^{T} \implies y^{*^{T}} = v P^{-1} u^{T} \implies y^{*^{T}} = \frac{P^{-1} u^{T}}{u P^{-1} u^{T}}$.

\paragraph{Лемма о масштабе} Пусть имеются две матричные игры с матрицами $P_{A}$ и $P_{A}'$ такие, что $P_{A}'$ получается из матрицы $P_{A}$ в виде линейного преобразования $P_{A}' = \alpha P_{A} + \beta$ ($\alpha$ и $\beta$ — числа), тогда в этих играх множество оптимальных стратегий игроков $A$ и $B$ совпадают. А $v_{A}' = \alpha v_{A} + \beta$.

Рассмотрим $P = \begin{pmatrix}
    1 & -1 \\
    -1 & 1
\end{pmatrix}$, $x^{*} = (\frac{1}{2}; \frac{1}{2})$, $y^{*} = (\frac{1}{2}; \frac{1}{2})$, $v = 0$. Ее определитель равен нулю. Используя лемму о масштабе, заменим матрицу $P$ на матрицу $P' = \frac{1}{2} (P + 1) = \begin{pmatrix}
    1 & 0 \\
    0 & 1
\end{pmatrix}$. Теперь мы имеем единичную матрицу, $(P')^{-1} = \begin{pmatrix}
    1 & 0 \\
    0 & 1
\end{pmatrix}$, $u(P)^{-1} = (1, 1, 1)$, $(P')^{-1} u^{T} = (1, 1)^{T}$, $u(P')^{-1} u^{T} = 1 + 1 = 2 \implies v' = \frac{1}{2}$, $x^{*} = \frac{1}{2} * (1, 1)$, $y^{*} = \frac{1}{2} (1, 1)$, $v' = \frac{1}{2} (v + 1) = \frac{1}{2} \implies v = 0$.


\subsubsection{Примеры}

\paragraph{Пример №1} Пусть $P = \begin{pmatrix}
    1 & 3 & 4 \\
    2 & 2 & 1 \\
    2 & 1 & 6
\end{pmatrix}$. В этой матрице нет доминируемых стратегий, то есть, ничего сократить невозможно. Кроме того, данная матрица имеет обратную: $|P| \ne 0$. Это позволяет применить вышенаписанные способы решения.

$P^{-1} = \frac{1}{27} \begin{pmatrix}
    -11 & 14 & 5 \\
    10 & 2 & -7 \\
    2 & -5 & 4
\end{pmatrix}$. Далее вычислим $u P^{-1} = (-11 + 10 + 2, 14 + 2 - 5, 5 - 7 + 4) = \frac{1}{27} (1, 11, 2)$. Далее: $u P^{-1} v^{T} = \frac{1}{27} (1 + 11 + 2) = \frac{14}{27} \implies v_{A} = \frac{1}{u P^{-1} u^{T}} = \frac{27}{14}$. Тогда, соответственно, $x^{*} = v u P^{-1} = \frac{27}{14} * \frac{1}{27} (1, 11, 2) = (\frac{1}{14}, \frac{11}{14}, \frac{2}{14})$. Найдем $y^{*}$: $P^{-1} u^{T} = \frac{1}{27} (-11 + 14 + 5, 10 + 2 - 7, 2 - 5 + 4) = \frac{1}{27} (8, 5, 1)$, $y^{*} = v P^{-1} u^{T} = \frac{1}{14} (8, 5, 1) = (\frac{8}{14}, \frac{5}{14}, \frac{1}{14})$.

\end{document}