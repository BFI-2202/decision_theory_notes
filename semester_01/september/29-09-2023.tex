\documentclass{article}
\usepackage[utf8]{inputenc}

\usepackage[T2A]{fontenc}
\usepackage[utf8]{inputenc}
\usepackage[russian]{babel}

\usepackage{tabularx}
\usepackage{amsmath}
\usepackage{pgfplots}
\usepackage{geometry}
\geometry{
    left=1cm,right=1cm,top=2cm,bottom=2cm
}
\newcommand*\diff{\mathop{}\!\mathrm{d}}

\newtheorem{definition}{Определение}
\newtheorem{theorem}{Теорема}

\DeclareMathOperator{\sign}{sign}

\usepackage{hyperref}
\hypersetup{
    colorlinks, citecolor=black, filecolor=black, linkcolor=black, urlcolor=black
}

\title{Теория принятия решений}
\author{Лисид Лаконский}
\date{September 2023}

\begin{document}
\raggedright

\maketitle

\tableofcontents
\pagebreak

\section{Практическое занятие — 29.09.2023}

На следующем занятии будет контрольная работа по пройденным темам: всякие разные матричные игры, см. остальные конспекты.

\subsection{Решение в чистых стратегиях}

\paragraph{Пример №1} Пусть $P_{A} = \begin{pmatrix}
    6 & 1 & 5 & 4 & 2 & 0 \\
    3 & 2 & 4 & 2 & 2 & 5 \\
    5 & 1 & 2 & 4 & 1 & 6 \\
    6 & 2 & 3 & 5 & 2 & 3 \\
    5 & 2 & 5 & 0 & 1 & 6
\end{pmatrix}$. Покажем решение в чистых стратегиях для данной матричной игры. Найдем максимин (максимум среди всех наименьших чисел в строках) и минимакс (минимум среди всех наибольших чисел в столбцах).

$\underline{v} = max \{ 0, 2, 1, 2, 0 \} = 2$

$\overline{v} = min \{ 6, 2, 5, 5, 2, 6 \} = 2$

Так как $\underline{v} = \overline{v} = v = 2$, то имеются седловые точки, необходимо их найти. У игрока А оптимальными стратегиями являются $A_{2}$, $A_{4}$. У игрока B оптимальными стратегиями являются $B_{2}$ и $B_{5}$. Таким образом, мы имеем \textbf{четыре седловые точки (ситуации равновесия)}: $(A_{2}, B_{2})$, $(A_{2}, B_{5})$, $(A_{4}, B_{2})$, $(A_{4}, B_{5})$.

\subsection{Решение в смешанных стратегиях}

\paragraph{Пример №1} Пусть $P_{A} = \begin{pmatrix}
    -6 & 1 & 4 & 2\\
    1 & 0 & -1 & 2 \\
    -1 & -1 & 2 & 0\\
    1 & 1 & -1 & 0\\
    1 & 2 & -1 & 1\\
    2 & 2 & -4 & 2
\end{pmatrix}$. Найдем максимин (максимум среди всех наименьших чисел в строках) и минимакс (минимум среди всех наибольших чисел в столбцах).

$\underline{v} = max \{ -6, -1, -1, -1, -1, -4 \} = -1$

$\overline{v} = min \{ 2, 2, 4, 2 \} = 2$

Так как $\underline{v} < \overline{v}$, наш выигрыш заключен на интервале от $-1$ до $2$. Решение в чистых стратегиях невозможно. Кроме того, необходимо сократить размерность данной матрицы, если это возможно. Исходя из принципов доминирования, можем вычеркнуть четвертую строку: $A_{4} \le A_{5}$

Таким образом, $P_{A} = \begin{pmatrix}
    -6 & 1 & 4 & 2\\
    1 & 0 & -1 & 2 \\
    -1 & -1 & 2 & 0\\
    1 & 2 & -1 & 1\\
    2 & 2 & -4 & 2
\end{pmatrix}$. Первый столбец доминирует четвертый столбец, так что $P_{A} = \begin{pmatrix}
    B_{1} & B_{2} & B_{3} \\
    -6 & 1 & 4 & A_{1}\\
    1 & 0 & -1 & A_{2}\\
    -1 & -1 & 2 & A_{3}\\
    1 & 2 & -1 & A_{5}\\
    2 & 2 & -4 & A_{6}
\end{pmatrix}$. Вторая строка меньше четвертой: $P_{A} = \begin{pmatrix}
    B_{1} & B_{2} & B_{3} \\
    -6 & 1 & 4 & A_{1}\\
    -1 & -1 & 2 & A_{3}\\
    1 & 2 & -1 & A_{5}\\
    2 & 2 & -4 & A_{6}
\end{pmatrix}$. Первый столбец доминирует второй, так что $P_{A} = \begin{pmatrix}
    B_{1} & B_{3} \\
    -6 & 4 & A_{1}\\
    -1 & 2 & A_{3}\\
    1 & -1 & A_{5}\\
    2 & -4 & A_{6}
\end{pmatrix}$. Это окончательный вариант матрицы. Больше доминирующих строк или столбцов нет. Имеем тип игры $m \times 2$.

Игрок $A$ имеет вектор смешанных стратегий: $x = \{ p_{1}, 0, p_{3}, 0, p_{5}, p_{6} \}$. Игрок $B$ имеет вектор смешанных стратегий: $y = \{ q_{1}, 0, q_{3}, 0 \} = \{ q, 0, 1 - q, 0 \}$. Задача решается с точки зрения игрока $B$ на плоскости $QoV$. Имеем отрезки:

$(1): v = -6y = 4(1 - q)$

$(3): v = -y + 2(1-q)$

$(5): v = 1*y - 1*(1-q)$

$(6): v = 2y - 4(1-q)$

Слева откладываем $B_{3}$, справа откладываем $B_{1}$. На этом рисунке мы должны выделить \textbf{верхнюю огибающую} и выделить на ней \textbf{нижнюю точку}.

$N$ — нижняя точка верхней огибающей — есть точка пересечения третьей и пятой прямых.

Решение для игрока $B$: $\begin{cases}
    v = -y + 2(1-q) \\
    v = 1*y - 1*(1-q)
\end{cases} \implies v_{A}^{*} = -3q^{*} + 2 = 2q^{*} - 1 \implies q^{*} = \frac{3}{5}, v_{A} = \frac{1}{5}$.

Решение для игрока $A$: $p_{1} = p_{6} = 0$, $p_{3} = p$, $p_{5} = 1 - p, \begin{cases}
    v = -1p + 1 (1 - p) \\
    v = 2p - 1 (1 - p)
\end{cases} \implies v_{A} = -2p^{*} + 1 = 3p^{*} -1 \implies p^{*} = \frac{2}{5}, v_{A} = \frac{1}{5}$.

Запишем наше решение в общем виде: $x^{*} = ( 0, 0, \frac{2}{5}, 0, \frac{3}{5}, 0 )$, $y^{*} = ( \frac{3}{5}, 0, \frac{2}{5}, 0 )$, $v_{A} = \frac{1}{5}$

\paragraph{Пример №2 (Игра полковника Блотто)} Полковнику Блотто (игрок А) поставлена задача: прорваться $m$ полками через 2 горных перевала, охраняемых $n$ полками противника (игрок B). Выигрыш полковника равен общему числу его полков, прорвавшихся через 2 перевала. Тогда выигрыш можно записать в следующем виде: $max (k_{1} - l_{1}, 0) + max(k_{2} - l_{2}, 0)$, где $k_{i}$ — число полков $A$ на $i$-ом перевале, $l_{i}$ — число полков $B$ на $i$-ом перевале. Составим матрицу платежей. Рассмотрим случай, когда $m = 3$, $n = 2$: $P_{A} = \begin{pmatrix}
    & (2, 0) & (1, 1) & (0, 2) \\
    (3, 0) & 1 & 2 & 3 \\
    (2, 1) & 1 & 1 & 2 \\
    (1, 2) & 2 & 1 & 1 \\
    (0, 2) & 3 & 2 & 1
\end{pmatrix} \implies P_{A} = \begin{pmatrix}
    1 & 2 & 3 \\
    1 & 1 & 2 \\
    2 & 1 & 1 \\
    3 & 2 & 1
\end{pmatrix}$. Посмотрим на эту игру с точки зрения доминирования: первая строка доминирует вторую, а четвертая доминирует третью строку: $P_{A} = \begin{pmatrix}
    1 & 2 & 3 & A_{1} \\
    3 & 2 & 1 & A_{4}
\end{pmatrix}$. Больше отношений доминирования не наблюдается.

Игрок $A$ имеет вектор смешанных стратегий: $x = \{ p, 0, 0, 1 - p \}$. Игрок $B$ имеет вектор смешанных стратегий: $y = \{ q_{1}, q_{2}, q_{3} \}$. Задача решается с точки зрения игрока $A$ на плоскости $PoV$. Имеем отрезки.

На этом рисунке мы должны выделить \textbf{нижнюю огибающую} и выделить на ней \textbf{верхнюю точку}.

$N$ — верхняя точка нижней огибающей — есть точка пересечения первой и третьей прямых.

Решение для игрока $A$: $\begin{cases}
    v = 1 * p + 3 (1 - p) \\
    v = 3 * p + 1 ( 1 - p)
\end{cases}$.

Решение для игрока $B$: $\begin{cases}
    v = 1 * q + 3 (1 - q) \\
    v = 3 * q + 1 (1 - q)
\end{cases}$, где $q = q_{1}, q_{3} = 1 - q$. Имеем: $p^{*} = q^{*} = \frac{1}{2}$, $v_{A} = 2$. Таким образом: $x^{*} = ( \frac{1}{2}, 0, 0, \frac{1}{2} )$, $y^{*} = ( \frac{1}{2}, 0, \frac{1}{2} )$

\paragraph{Пример №3} Пусть $P = \begin{pmatrix}
    -3 & 1 & 2 \\
    2 & 0 & -1 \\
    1 & 0 & 2 
\end{pmatrix}$. Найдем максимин (максимум среди всех наименьших чисел в строках) и минимакс (минимум среди всех наибольших чисел в столбцах).

$\underline{v} = max \{ -3, -1, 0 \} = 0$

$\overline{v} = min \{ 2, 1, 2 \} = 1$

Так как $\underline{v} < \overline{v}$, наш выигрыш заключен на интервале от $0$ до $1$. Решение в чистых стратегиях невозможно. Кроме того, необходимо сократить размерность данной матрицы, если это возможно. В матрице нет доминируемых строк или столбцов. Можем применить метод обратной матрицы.

Если $|P| \ne 0$, то $x^{*} = \frac{u P^{-1}}{u P^{-1} u^{T}}$, $y^{*} = \frac{P^{-1} u^{T}}{u P^{-1} u^{T}}$, $v = \frac{1}{u P^{-1} u^{T}}$

Найдем обратную матрицу: $\begin{pmatrix}
    -3 & 1 & 2 & 1 & 0 & 0 \\
    2 & 0 & -1 & 0 & 1 & 0 \\
    1 & 0 & 2 & 0 & 0 & 1
\end{pmatrix} \sim \begin{pmatrix}
    1 & 0 & 2 & 0 & 0 & 1 \\
    2 & 0 & -1 & 0 & 1 & 0 \\
    -3 & 1 & 2 & 1 & 0 & 0
\end{pmatrix} \sim \begin{pmatrix}
    1 & 0 & 2 & 0 & 0 & 1 \\
    0 & 0 & -5 & 0 & 1 & -2 \\
    0 & 1 & 8 & 1 & 0 & 3
\end{pmatrix} \sim \begin{pmatrix}
    1 & 0 & 2 & 0 & 0 & 1 \\
    0 & 1 & 8 & 1 & 0 & 3 \\
    0 & 0 & -5 & 0 & 1 & -2
\end{pmatrix} \sim \begin{pmatrix}
    1 & 0 & 2 & 0 & 0 & 1 \\
    0 & 1 & 8 & 1 & 0 & 3 \\
    0 & 0 & 1 & 0 & \frac{-1}{5} & \frac{2}{5}
\end{pmatrix} \sim \begin{pmatrix}
    1 & 0 & 0 & 0 & \frac{2}{5} & \frac{1}{5} \\
    0 & 1 & 0 & 1 & \frac{8}{5} & -\frac{1}{5} \\
    0 & 0 & 1 & 0 & -\frac{1}{5} & \frac{2}{5}
\end{pmatrix}$

Таким образом, $P^{-1} = -\frac{1}{5} \begin{pmatrix}
    0 & -2 & -1 \\
    -5 & -8 & 1 \\
    0 & 1 & -2
\end{pmatrix}$

$u = (1, 1, 1)$, $u P^{-1} = -\frac{1}{5} (0 - 5 + 0, -2 - 8 + 1, -1 + 1 - 2) = -\frac{1}{5} (-5, -9, -2) = \frac{1}{5} (5, 9, 2)$, $P^{-1} u^{T} = -\frac{1}{5} (0 - 2 - 1, -5 - 8 + 1, 0 + 1 - 2)^{T} = \frac{1}{5} (3,12,1)^{T}$, $u P^{-1} u^{T} = \frac{1}{5} (5 + 3 + 2) = \frac{1}{5} (3 + 12 + 1) = \frac{16}{5}$, $v_{A} = \frac{5}{16}$, $x^{*} = v_{A} u P^{-1} = \frac{1}{16} (5, 3, 2) = (\frac{5}{16}, \frac{9}{16}, \frac{2}{16})$, $y^{*} = v_{A} (P^{-1} u^{T})^{T} = \frac{1}{16} (3, 12, 1) = (\frac{3}{16}, \frac{12}{16}, \frac{1}{16})$.

Если бы матрица была бы вырожденной ($|P| = 0$), то нам необходимо было бы перейти к матрице $P' = \alpha P + \beta$, тогда $v' = \alpha * v + \beta$, $x' = x$, $y' = y$

\paragraph{Пример №4}

$P = \begin{pmatrix}
    1 & -1 \\
    -1 & 1
\end{pmatrix}, |P| = 0, P' = \frac{1}{2} (P + 1) = \begin{pmatrix}
    1 & 0 \\
    0 & 1
\end{pmatrix}$

\paragraph{Пример №5}

$P = \begin{pmatrix}
    1 & 0 & -1 \\
    0 & 2 & 1 \\
    -1 & 0 & 2 
\end{pmatrix}, \underline{v} = 0, \overline{v} = 1$. Кроме того, доминируемых стратегий также не имеется. Ищем решение через обратную матрицу:

$P^{-1} = \begin{pmatrix}
    2 & 0 & 1 \\
    -\frac{1}{2} & \frac{1}{2} & -\frac{1}{2} \\
    1 & 0 & 1
\end{pmatrix}$

По формулам получается: $u P^{-1} = (2 - \frac{1}{2} + 1, 0 + \frac{1}{2} + 0, 1 - \frac{1}{2} + 1) = ( \frac{5}{2}, \frac{1}{2}, \frac{3}{2} )$, $P^{-1} u^{T} = (3, -\frac{1}{2}, 2)$, $u P^{-1} u^{T} = \frac{5}{2} + \frac{1}{2} + \frac{3}{2} = \frac{9}{2} = 3 - \frac{1}{2} + 2 = \frac{9}{2} \implies v_{A} = \frac{2}{9}$, $x^{*} = ( \frac{5}{9}, \frac{1}{9}, \frac{3}{9} )$, $y^{*} = \frac{2}{9} (3, -\frac{1}{2}, 2) = ( \frac{6}{9}, -\frac{1}{9}, \frac{4}{9})$. Получили отрицательное число — отрицательную вероятность. Значит, в этом решении что-то не так. Этот вектор показывает, что игроку $B$ необходимо отказаться от второй стратегии. Так как $q_{2}^{*} < 0$, то $q_{2}^{*} = 0 \implies P' = \begin{pmatrix}
    1 & -1 \\
    0 & 1 \\
    -1 & 2
\end{pmatrix}$. Дальше эту игру надо решать как игру $3 \times 2$.

\subsection{Домашнее задание}

Решить предыдущий пример, но правильно.

\end{document}