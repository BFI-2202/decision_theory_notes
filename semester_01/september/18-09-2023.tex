\documentclass{article}
\usepackage[utf8]{inputenc}

\usepackage[T2A]{fontenc}
\usepackage[utf8]{inputenc}
\usepackage[russian]{babel}

\usepackage{tabularx}
\usepackage{amsmath}
\usepackage{pgfplots}
\usepackage{geometry}
\geometry{
    left=1cm,right=1cm,top=2cm,bottom=2cm
}
\newcommand*\diff{\mathop{}\!\mathrm{d}}

\newtheorem{definition}{Определение}
\newtheorem{theorem}{Теорема}

\DeclareMathOperator{\sign}{sign}

\usepackage{hyperref}
\hypersetup{
    colorlinks, citecolor=black, filecolor=black, linkcolor=black, urlcolor=black
}

\title{Теория принятия решений}
\author{Лисид Лаконский}
\date{September 2023}

\begin{document}
\raggedright

\maketitle

\tableofcontents
\pagebreak

\section{Практическое занятие — 18.09.2023}

\definition{Лемма о масштабе}. Говорит о стратегической эквивалентности двух игр, отличающихся только масштабом измерений. Пусть имеется две матричные игры, каждая из которых задается своей матрицей: $(P_{A}, v_{A})$, $(P_{A}', v_{A}')$, причем $P_{A}' = \alpha * P_{A} + \beta$ ($\alpha, \beta$ — числа), тогда оптимальные векторы стратегий в этих играх совпадают: $x^{*} = (x')^{*}$, $y^{*} = (y')^{*}$, а цена игры изменится: $v_{A}' = \alpha * v_{A} + \beta$.

Пример: $P = \begin{pmatrix}
	1 & -1 \\
	-1 & 1
\end{pmatrix} \implies P' = \begin{pmatrix}
	1 & 0 \\
	0 & 1
\end{pmatrix}$. Эту матрицу $P'$ можно получить следующим образом: $P + 1 = \begin{pmatrix}
	2 & 0 \\
	0 & 2
\end{pmatrix} \implies P' = \frac{1}{2}(P + 1)$

Аналитическое решение для матрицы $P' = \begin{pmatrix}
	1 & 0 \\
	 0 & 1
\end{pmatrix}$:

\begin{enumerate}
	\item $A$: $x' = (p', 1 - p')$

	$1 * p' + 0 (1 - p') = 0 * p' + 1 (1 - p') \implies p' = 1 - p' \implies (p')^{*} = \frac{1}{2}$

	$(x')^{*} = (\frac{1}{2}; \frac{1}{2})$

	$v' = (p')^{*} = \frac{1}{2}$

	$v' = \frac{1}{2}(v + 1) = \frac{1}{2} \implies v = 0$
\end{enumerate}

\subsection{Геометрические методы решения матричных игр}

\subsubsection{Решение матричной $2 \times n$ игры}

$P = \begin{pmatrix}
	a_{11} & a_{12} & \dots & a_{1n} \\
	a_{21} & a_{22} & \dots & a_{2n}
\end{pmatrix}$

Рассмотрим $PoV (x = p, y = v)$, $x = (p, 1 - p)$ — смешанная стратегия игрока А, $y = (q_{1}, \dots, q_{n}$ — смешанная стратегия игрока B, $\sum\limits_{j = 1}^{n} q_{j} = 1$.

Тогда цена игры $v = max_{p} min_{j} H_{A}(p, j)$, где $H_{A}(p, j) = \begin{pmatrix} p & 1 - p \end{pmatrix} \begin{pmatrix} a_{1j} \\ a_{2j} \end{pmatrix} = a_{ij} * p + a_{2j} (1 - p)$.

Рассмотрим прямые линии $v = a_{1j} * p + a_{2j} (1 - p)$, $p = p^{*}$.

Для того, чтобы определить, какие линии дают оптимальное решение, нужно построить уравнение этих прямых на плоскости и найти их минимум: $min_{j} \{ a_{1j}p + a_{2j}(1 - p) \}$ — нижняя огибающая семейства прямых. На графике отмечаем самые нижние части всех прямых линий. Далее на этой нижней огибающей необходимо найти наивысшую точку (то есть, максимум). Именно эта точка дает решение: $M(p^{*}, v^{*})$, $v^{*} = v_{A} = v$ — цена игры. Точка $M$ является точкой пересечения двух участков нижней огибающей, которые показывают, какие стратегии выбрал игрок B.

Пусть прямые $H_{A}(p, k)$ и $H_{A}(p, l)$ дают точку пересечения. Тогда, чтобы найти решение, их нужно приравнять между собой:

$v = H_{A}(p, k) = H_{A}(p, l)$, $p = p^{*}$. Или, если записать в общем виде: $v = a_{1k} * p + a_{2k} (1 - p) = a_{1l}p + a_{2l}(1 - p)$, $p=p^{*}$

\textbf{Решение для B}

$q_j = 0, j \ne k,l$. $q_{k} = q$, $q_{l} = 1 - q$

$v = a_{1k} q + a_{1l} (1 - q) = a_{2k} q + a_{2l} (1 - q)$, $q=q^{*}$

\paragraph{Частные случаи решений в зависимости от формы нижней огибающей}

\begin{enumerate}
	\item Нижняя огибающая имеет одну наивысшую точку $M(p^{*}, v^{*})$
	\begin{enumerate}
		\item $p^{*} = 0$. Точка $M$ не является точкой пересечения двух участков нижней огибающей, как было записано ранее. Она лежит на какой-то $j$-ой прямой. Значит, игрок B в качестве оптимальной выбирает $j$-ую стратегию.
		\item $p^{*} = 1$. Ситуация, симметричная предыдущему пункту. Игрок $A$ выбирает стратегию $A_{1}$. Какую стратегию выбирает игрок $B$? Ту, на которую лежит точка $M$, то есть, $B_{j}$, соответствующую $j$-ой прямой, на которой лежит точка $M$.
		\item $0 \le p^{*} \le 1$. Решение смотри выше: где мы рассматривали самое первое решение матричной $2 \times n$ игры на оси $PoV$.
	\end{enumerate}
	\item Нижняя огибающая содержит горизонтальные участки, соответствующие $j$-ой строке игрока $B$. В этом случае решение в смешанных стратегиях отсутствует, есть решение только в чистых стратегиях. Так как игрок $B$ выбирает какую-то $j$-ую стратегию, то это столбец $\begin{pmatrix} a_{1j} \\ a_{2j} \end{pmatrix}$. Тогда игрок $A$ выберет такую свою чистую стратегию, при которой его выигрыш максимален.

В данном случае не надо искать точку пересечения прямых на плоскости. Решаем в чистых стратегиях, как мы все давно прекрасно умеем делать.
\end{enumerate}

\paragraph{Примеры}

Пусть $P{A} = \begin{pmatrix}
	2 & -3 & 5 \\
	4 &  5 & -1
\end{pmatrix}$. Игрок $A$ имеет вектор стратегий: $x = (p, 1 - p)$, игрок $B$: $y = (q_1, q_2, q_3)$. Записываем по столбцам матрицы:

$H_{A}(p, j) = v, j = 1,2,3$

$v = 2p+4(1-p)$ — для $B_1$

$v = -3p + 5(1-p)$ — для $B_2$

$v = 5p-1(1-p)$ — для $B_3$

Далее изобразим это на графике. Каждую пару точек соединим прямыми линиями: $2$ на оси $A_{1}$ и $4$ на оси $A_{2}$, далее $-3$ на оси $A_{1}$ и $5$ на оси $A_{2}$, $5$ на оси $A_{1}$ и $-1$ на оси $A_{2}$. Далее найдем наименьшую точку нижней огибающей. Она содержит в себе только два участка прямых: линии $B_{2}$ и $B_{3}$. По рисунку точка $M$ является пересечением второй и третьей линии.

$M$ — точка пересечения второй и третьей прямых:

$v = -3p+5(1-p) = 5p-(1-p), p=p^{*} \implies 5 - 8p^{*} = 6p^{*}-1 \implies p^{*} = \frac{6}{14} = \frac{3}{7}$.

Цена игры: $v = 5 - 8 * \frac{3}{7} = \frac{35 - 24}{7} = \frac{11}{7}$

Таким образом, $x^{*} = (\frac{3}{7}; \frac{4}{7})$.

Найдем вектор $y$: $q_{1} = 0$, $q_{2} = q$, $q_{3} = 1 - q$.

$v = -3q + 5(1-q) = 2 * 0 + (-3) * q + 5 (1 - q)$

$v = 5q - 1 * (1 - q), q = q^{*}$

Отсюда: $5 - 8q^{*} = 6q^{*} - 1 \implies q^{*} = \frac{3}{7}$.

Таким образом, вектор $q^{*}  = (0; \frac{3}{7}; \frac{4}{7})$

\subsubsection{Решение матричной $m \times 2$ игры}

Платёжная матрица имеет следующий вид: $P = \begin{pmatrix}
	a_{11} & a_{12} \\
	a_{21} & a_{22} \\
	\dots & \dots \\
	a_{m1} & a_{m2}
\end{pmatrix}$

Решение этой игры аналогично предыдущему случаю, но оно графически строится уже с точки зрения игрока $B$. Он имеет две стратегии: $y = (q, 1-q), 0 \le q \le 1$. Игрок А имеет вектор стратегий: $x = (p_1, p_2, \dots, p_n), 0 \le p_{i} \le 1, \sum\limits_{i = 1}^{m} p_{i} = 1$.

$v^{*} = v_{A} = v = min_{q} max_{1 \le i \le m} H_{A}(i, q)$

$H_{A}(i, q) = a_{i1} * q + a_{i2} * (1 - q)$

На плоскости $qOv$ построим семейство прямых линий $v = H_{A}(i, q)$, $1 \le i \le m$. Далее на множестве этих прямых выделяем верхнюю огибающую и на этой верхней огибающей в качестве точки оптимума находим нижнюю точку $N(q^{*}, v^{*})$, в которой и будет достигаться решение.

$v^{*} = a_{k1} q^{*} + q_{k2} * (1 - q^{*}) = a_{l1} q^{*} + a_{l2} (1 - q^{*})$, где $k$, $l$ — линии, составляющие верхнюю огибающую.

\textbf{Решение для А}

$p_{i} = 0, i \ne k, l; p_k = p_, p_{l} = 1 - p$

$P' = \begin{pmatrix}
	a_{k1} & a_{k2} \\
	a_{l1} & a_{l2}
\end{pmatrix}$

$v^{*} = a_{k1} * p^{*} + a_{l1} (1 - p^{*}) = a_{k2} p^{*} + a_{l2} (1 - p^{*})$

\paragraph{Пример №1} Необходимо найти решение в смешанных стратегиях для следующей матрицы:

$P = \begin{pmatrix}
	3 & -1 \\
	2 & 4 \\
	1 & 0
\end{pmatrix}$

Игрок А: $x = (p_1, p_2, \dots p_n)$

Игрок B: $y = (q, 1-q)$

$A_{1}: v = 3q - 1(1-q)$

$A_{2}: v = -2q + 4(1 - q)$

$A_{3}: v = 1 * q + 0 (1 - q)$

Построим график, соединим точки. Найдем верхнюю огибающую и точку $N(q^{*}, v^{*})$. По графику видно, что мы смешиваем первую и вторую стратегию, третья стратегия не является выгодной дли игрока А.

Решение сводится к рассмотренному ранее на предыдущей паре.

\textbf{Решение для B}:

$3q - (1 - q) = -2 q + 4 (1 - q) = v, q = q^{*}$

$4q^{*} - 1 = 4 - 6y^{*} \implies q^{*} = \frac{5}{10} = \frac{1}{2}$

$v = 4 * \frac{1}{2} - 1 = 1$

\textbf{Решение для A}. Не забываем, что оно строится не по строкам матрицы, а по столбцам. Хотя строки являются его стратегиями, комбинациии строятся по столбцам:

$p_{3} = 0, p_{2} = 1 - p, p_{1} = p$.

$v = 3p - 2 (1 - p) = -p + 4(1 - p), p = p^{*} \implies 5p^{*} - 2 = 4 - 5p^{*} \implies p^{*} = \frac{6}{10} = \frac{3}{5} \implies x^{*} = ( \frac{3}{5}, \frac{2}{5}, 0), y^{0} = (\frac{1}{2}, \frac{1}{2}), v = 5 * \frac{3}{5} - 2 = 3 - 2 = 1$

\subsection{Домашнее задание}

\paragraph{Задание №1} $P = \begin{pmatrix}
	1 & 3 & 1 & 4 \\
	2 & 1 & 4 & 0
\end{pmatrix}$

\paragraph{Задание №2} $P = \begin{pmatrix}
	6 & 4 & 3 & 1 & -1 & 0 \\
	-2 & -1 & 1 & 0 & 5 & 4
\end{pmatrix}$

\paragraph{Задание №3} $P = \begin{pmatrix}
	3 & 4 \\
	5 & 1 \\
	2 & 5
\end{pmatrix}$

\paragraph{Задание №4} $P = \begin{pmatrix}
	2 & -4 \\
	-2 & 5 \\
	0 & 2
\end{pmatrix}$

\end{document}