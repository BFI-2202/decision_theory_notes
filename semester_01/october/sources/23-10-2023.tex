\documentclass{article}
\usepackage[utf8]{inputenc}

\usepackage[T2A]{fontenc}
\usepackage[utf8]{inputenc}
\usepackage[russian]{babel}

\usepackage{tabularx}
\usepackage{amsmath}
\usepackage{pgfplots}
\usepackage{geometry}
\geometry{
    left=1cm,right=1cm,top=2cm,bottom=2cm
}
\newcommand*\diff{\mathop{}\!\mathrm{d}}

\newtheorem{definition}{Определение}
\newtheorem{theorem}{Теорема}

\DeclareMathOperator{\sign}{sign}

\usepackage{hyperref}
\hypersetup{
    colorlinks, citecolor=black, filecolor=black, linkcolor=black, urlcolor=black
}

\tikzset{
  dot/.style={circle, fill, inner sep=1.5pt},
  marked/.style={
    postaction={decorate},
    decoration={
        markings,
        mark=
        between positions 0 and 1 step 3mm
        with
        {
            \draw[thick, color=gray!70, shorten >= 0, shorten <= 0, -] (0,0) -- (0.1,0.2);
        }
    }
  },
  markedP/.style={
    postaction={decorate},
    decoration={
        markings,
        mark=
        between positions 0.05 and 0.95 step 8mm
        with
        {
            \draw[thick, shorten >= 0, shorten <= 0, -] (0,-0.5) -- (0,0.5);
        }
    }
  },
  extend/.style={shorten >= -#1, shorten <= -#1},
  extend/.default=0.7cm
}
\usetikzlibrary{decorations.markings, patterns, calc}

\title{Теория принятия решений}
\author{Лисид Лаконский}
\date{October 2023}

\begin{document}
\raggedright

\maketitle

\tableofcontents
\pagebreak

\section{Лекция — 23.10.2023}

Докажем следующее утверждение: множество всех допустимых решений системы ограничений задачи линейного программирования является выпуклым множеством.

$F(\overline{x}) = \sum\limits_{i = 1}^{n} C_{i} x_{i} \to max (min)$

$AX = B$, $A = (a_{i j})_{m \times n}$, $X = (x_1, \dots, x_{n})^{T}$, $B = (b_1, \dots, b_{m})^{T}$

Пусть $X_{1}^{*}$, $X_{2}^{*}$ — решение системы ограничений, то есть, $A X_{1}^{*} = A X_{2}^{*} = B$

Покажем, что $X^{*} = \alpha X_{1}^{*} + (1 - \alpha) X_{2}^{*}$, где $0 \le \alpha \le 1$ также является решением системы ограничений.

$AX^{*} = A(\alpha X_{1}^{*} + (1 - \alpha) X_{2}^{*}) = \alpha * AX_{1}^{*} + (1 - \alpha) * AX_{2}^{*} = \alpha B + (1 - \alpha) B - B$

Так как $X^{*}$ — это выпуклое множество...

\subsection{Решение задач линейного программирования}

\begin{theorem}
Если задача линейного программирования (далее — ЗЛП) имеет оптимальное решение, то линейная функция $F(\overline{x})$ \textbf{достигает своего оптимума (то есть, максимума или минимума) в одной из угловых точек многогранника (многоугольника на плоскости) решений}.

Отметим, что \textbf{если оптимальное решение достигается более чем в одной точке}, то \textbf{оно является выпуклой линейной комбинацией этих точек}.
\end{theorem}

\begin{theorem}
    Каждому допустимому базисному решению $X^{*} = (x_1^*, x_2^*, x_n^*)$ ЗЛП \textbf{соответствует одна из угловых точек многогранника (многоугольника на плоскости) решений}.

    Таким образом, \textbf{любое оптимальное решение ЗЛП является одной из угловых точек}, то есть \textbf{допустимым базисным решением}.
\end{theorem}

\subsubsection{Геометрический способ решения}

Рассмотрим решение на плоскости. Оно возможно в следующих случаях:

\begin{enumerate}
    \item Имеем две переменные: $n = 2$
    \item Количество неизвестных минус количество уравнений равно двум: $n - m = 2$
\end{enumerate}

Чтобы показать геометрическое решение, необходимо выбрать $m$ переменных из $x_1, \dots, x_{n}$ в качестве основных (главных, базисных), а остальные переменные назовем свободными.

Пусть $x_1$, $x_2$ — свободные переменные, а $x_3, \dots, x_{n}$ — базисные.

$$
a_{i 1} x_{1} + \dots + a_{i n} x_{n} = b_{i} \implies
$$

$$
\begin{cases}
x_3 = \beta_{3} + \alpha_{3 1} x_1 + \alpha_{3 2} x_{2} \\
\dots \\
x_n = \beta_{n} + \alpha_{n 1} x_1 + \alpha_{n 2} x_{2}
\end{cases}
$$

Так как $x_{i} \ge 0$, $\forall i = \overline{1, n} \implies \begin{cases}
    \beta_{i} + \alpha_{i 1} x_1 + \alpha_{i 2} x_2 \ge 0 \\
    x_1, x_2 \ge 0
\end{cases}$

\textbf{Нужно показать геометрически множество решений этой системы}, которое является выпуклым многоугольником.

\begin{enumerate}
    \item \textbf{Если нет пересечений в системе ограничений, то решение не существует}.
    \item \textbf{Если есть пересечение, то решение будет}.
\end{enumerate}

$F(\overline{x})$ выше было записано в следующем виде: $F(\overline{x}) = c_1 x_1 + c_2 x_2 + \dots + c_n x_n$. \textbf{Необходимо переписать ее, выразив базисные переменные через свободные в следующем виде}: $F(\overline{x}) = \gamma_1 x_1 + \gamma_2 x_2 + \gamma_{0}$

Если рассматривается задача на максимум, то \textbf{решение нужно искать в направлении возрастания функции} $F(\overline{x})$. Направление возрастания функции $F(\overline{x})$ — вектор градиент: $grad \ F = \{ \frac{\delta F}{\delta x_1}, \dots, \frac{\delta F}{\delta x_{n}} \}$. В нашем случае: $grad \ F = \{ \gamma_1, \gamma_2 \}$.

То есть, \textbf{необходимо нарисовать на плоскости $x_1 o x_2$ уравнение прямых} $\gamma_1 x_1 + \gamma_2 x_2 = c$ (опорные прямые), перпендикулярные $grad \ F$

\textbf{Если мы хотим найти максимум, то на многоугольнике решений надо найти точку, через которую проходит линия уровня (то есть, опорная прямая) с наибольшим значением уровня} (то есть, константы $c$). Получающаяся точка и является оптимальным решением.

В случае \textbf{задачи на минимум} необходимо двигаться в направлении $- grad \ F$.

\begin{figure}[ht]
    \centering
    \begin{tikzpicture}
        \draw[very thin,color=gray!50] (-0.5,-0.5) grid (5.9,4.9);
        \draw[->] (-0.5,0) -- (6.2,0) node[right] {$x_1$};
        \draw[->] (0,-0.5) -- (0,4.6) node[above] {$x_2$};

        \coordinate[dot, label=right:$A$] (A) at (0, 2);
        \coordinate[dot, label=above:$B$] (B) at (2, 0);
        \coordinate[dot, label=above:$C$] (C) at (6, 4);

        \draw [very thick, marked, extend] (A) -- (B);
        \draw [very thick, marked, extend] (B) -- (C);

        \draw[->, very thick, color=blue, markedP] (0,0) -- (5,5) node[above] {$\mathrm{grad} F$};

        \node[right, xshift=0.2cm] at (C) {$F = \max$};
    \end{tikzpicture}
    \caption{Чертёж к определению геометрического способа решения ЗЛП}
\end{figure}

\paragraph{Пример №1}

Решить геометрически задачу линейного программирования:

$$F(\overline{x}) = 2 x_1 + 3x_2 \to max$$

При ограничениях:

$$
\begin{cases}
  x_1 + 3x_2 \le 18 \implies \frac{x_1}{18} + \frac{x_2}{6} \le 1\\
  2x_1 + x_2 \le 16 \implies \frac{x_1}{8} + \frac{x_2}{16} \le 1\\
  x_2 \le 5\\
  x_1 \le 7\\
  x_1, x_2 \ge 0  
\end{cases}
$$

Нарисуем все эти ограничения на плоскости $x_1 o x_2$. В результате получаем выпуклый шестиугольник.

$gradient \ F = (2, 3)$

Опорная прмая: $2x_1 + 3x_2 = c$

Рассмотрим:

$$
\begin{cases}
    x_1 + 3x_2 = 18 \implies x_1 = 18 - 3x_2 \\
    2x_1 + x_2 = 16 \implies 36 - 6x_2 + x_2 = 16 \implies x_2 = 4 \implies x_1 = 6
\end{cases}
$$

Получаем решение: $X^{*} = (6, 4), F(X^{*}) = 24$

\begin{figure}[ht]
    \centering
    \begin{tikzpicture}[scale=0.50]
        \draw[very thin,color=gray!30] (0,0) grid (19.9,17.9);
        \draw[->] (-0.5,0) -- (20,0) node[right] {$x_1$};
        \draw[->] (0,-0.5) -- (0,18) node[above] {$x_2$};

        \coordinate[dot] (A) at (18, 0);
        \coordinate[dot] (B) at (0, 6);
        \coordinate[dot] (C) at (8, 0);
        \coordinate[dot] (D) at (0, 16);

        \draw [thick, marked, extend] (A) -- (B);
        \draw [thick, marked, extend] (C) -- (D);
        \draw [thick, marked, extend] (20, 5) -- (0, 5);
        \draw [thick, marked, extend] (7, 0) -- (7, 16);
        \draw [marked] (0, 17) -- (0,0);
        \draw [marked] (0, 0) -- (19,0);

        \draw [very thick, color=red, pattern = north west lines, pattern color = gray] (0, 0) -- (0, 5) -- (3, 5) -- (6, 4) -- (7, 2) -- (7, 0) -- cycle;

        \draw [thick, color=blue, extend=2cm] ($(0,0)!(6,4)!(2,3)$) -- (6,4);

        \draw[->, very thick, color=blue, markedP] (0,0) -- (8,12) node[above] {$\mathrm{grad} F$};

        \node[dot, color=red] at (6, 4) {};

        \foreach \point in {(0,0), (0,5), (3,5), (7,2), (7,0)}
            \node[dot] at \point {};
        \foreach \x in {1,...,6,8,9,...,18}
            \draw (\x, 1mm) -- (\x, -1mm) node[below] {$\x$};
        \foreach \y in {1,...,16}
            \draw (1mm, \y) -- (-1mm, \y) node[left] {$\y$};
        \foreach \x in {0,7}
            \draw (\x, 1mm) -- (\x, -1mm) node[below left] {$\x$};
    \end{tikzpicture}
    \caption{Чертёж к примеру 1}
\end{figure}

\paragraph{Пример №2}

Найти геометрически решение ЗЛП:

$$
F(\overline{x}) = 4x_1 - 3x_2 - x_4 + x_5 \to min
$$

При ограничениях:

$$
\begin{cases}
    -x_1 + 3x_2 + x_4 = 13 \\
    4x_1 + x_2 + x_5 = 26 \\
    -2x_1 + x_2 + x_3 = 1 \\
    x_1 - 3x_2 + x_6 = 0 \\
    x_{i} \ge 0, i = \overline{1, 6}    
\end{cases}
$$

Мы можем найти решение графически, так как $n = 6$, $m = 4$, $n - m = 2$. Из этой системы мы можем получить:

$$
\begin{cases}
  x_4 = 13 + x_{1} - 3x_{2} \ge 0 \\
  x_5 = 26 - 4x_1 - x_2 \ge 0 \\
  x_3 = 1 + 2x_1 - x_2 \ge 0 \\
  x_6 = -x_1 + 3x_2 \ge 0 \\
  x_1 x_2 \ge 0  
\end{cases} \implies
\begin{cases}
    x_1 - 3x_2 \ge -13 \\
    4x_1 + x_2 \le 26 \\
    x_2 \le 1 + 2x_1\\
    x_1 \le 3x_2 \\
    x_1 x_2 \ge 0
\end{cases}
$$

Нарисуем все эти ограничения на плоскости $x_1 o x_2$. Получаем многоугольник. Преобразуем $F(\overline{x})$, чтобы определить, какая из угловых точек является решением:

$$
F(\overline{x}) = 4x_1 - 3x_2 - (13 + x_1 - 3x_2) + (26 - 4x_1 - x_2) = -x_1 - x_2 + 13
$$

Таким образом:

$$
\gamma_1 = 1, \gamma_2 = 2, \gamma_3 = 13
$$

$$
gradient \ F = \{ -1, -1 \}, - gradient \ F = \{ 1, 1 \}
$$

Построим его на графике вместе с опорными линиями в его направлении. Видим: $A = \{ 5, 6 \}, F(A) = -5 - 6 + 13 = 2$. Проверим другие точки: $B = \{ 6, 2 \}, F(B) = -6 - 2 + 13 = 5$, $C = \{ 2, 5 \}, F(C) = -2 - 5 + 13 = 6$. Окончательно убеждаемся: $A$ — т. $min$

$x_4 = x_5 = 0$, $x_3 = 1 + 2 * 5 - 6 = 5$, $x_6 = -5 + 3 * 6 = 13 \implies X^{*} = (5, 6, 5, 0, 0, 13), F(X^{*}) = 2$

\begin{figure}[ht]
    \centering
    \begin{tikzpicture}
        \draw[very thin,color=gray!30] (0,0) grid (6.9,6.9);
        \draw[->] (-0.5,0) -- (7,0) node[right] {$x_1$};
        \draw[->] (0,-0.5) -- (0,7) node[above] {$x_2$};

        \coordinate[label={[above right]:$A$}] (A) at (5, 6);
        \coordinate[label={[below right]:$B$}] (B) at (6, 2);
        \coordinate[label={[above left]:$C$}] (C) at (2, 5);
        \coordinate[] (D) at (0, 1);

        \draw [thick, marked, extend=3cm] (A) node[above right, xshift=2cm]{$(1)$} -- (C);
        \draw [thick, marked, extend=1cm] (B) node[xshift=-0.2cm, yshift=-0.5cm]{$(2)$} -- (A);
        \draw [thick, marked, extend] (0,0) -- (B) node[right, xshift=0.5cm]{$(4)$};
        \draw [thick, marked, extend] (C) -- (D) node[left, yshift=-0.5cm, xshift=-0.2cm]{$(3)$};
        \draw [marked] (D) -- (0,0);

        \draw [very thick, color=red, pattern = north west lines, pattern color = gray] (0, 0) -- (D) -- (C) -- (A) -- (B) -- cycle;

        \draw [thick, color=blue, extend=2cm] ($(0,0)!(A)!(1,1)$) -- (A);

        \draw[->, very thick, color=blue, markedP] (0,0) -- (7,7) node[above] {$-\mathrm{grad} F$};

        \node[dot, color=red] at (A) {};

        \draw[dashed] (0, 5) -- (C);
        \draw[dashed] (0, 6) -- (A);
        \draw[dashed] (0, 2) -- (B);
        \draw[dashed] (2, 0) -- (C);
        \draw[dashed] (5, 0) -- (A);
        \draw[dashed] (6, 0) -- (B);

        \foreach \point in {(0,0), (B), (C), (D)}
            \node[dot] at \point {};
        \foreach \x in {1,...,6}
            \draw (\x, 1mm) -- (\x, -1mm) node[below] {$\x$};
        \foreach \y in {1,...,6}
            \draw (1mm, \y) -- (-1mm, \y) node[left] {$\y$};
        \draw (0, 1mm) -- (0, -1mm) node[below left] {$0$};
    \end{tikzpicture}
    \caption{Чертёж к примеру 2}
\end{figure}

\paragraph{Пример №3}

Найти геометрически решение ЗЛП:

$$
F(\overline{x}) = x_1 + 3x_2 + 3x_4 - x_3 \to max
$$

При ограничениях:

$$
\begin{cases}
    x_1 - 3x_2 + 3x_3 - 6x_4 = 0 \\
    3x_2 - 2x_3 + 6x_4 = 2 \\
    x_{i} \ge 0, i = \overline{1, 4}
\end{cases}
$$

Сложим вместе первое и второе уравнение:

$$
x_1 + x_3 = 2 \implies x_3 = 2 - x_1 \ge 0
$$

Сложим первое уравнение дважды и второе уравнение трижды:

$$
2x_1 + 3x_2 + 6x_4 = 6 \implies x_4 = 1 - \frac{1}{3} x_1 - \frac{1}{2}x_2 \ge 0
$$

Получили:

$$
\begin{cases}
    2 - x_1 \ge 0 \\
    \frac{x_1}{3} + \frac{x_2}{2} \le 1 \\
    x_1, x_2 \ge 0
\end{cases}
$$

Нарисуем все эти ограничения на плоскости $x_1 o x_2$. Получаем четырехугольник. Преобразуем $F(\overline{x})$, чтобы определить, какая из угловых точек является решением:

$F(\overline{x}) = x_1 + 3x_2 + 3 (1 - \frac{1}{3} x_1 - \frac{1}{2} x_2) - (2 - x_1) = x_1 + 3x_2 + 3 - x_1 - \frac{3x_2}{2} - 2 + x_1 = x_1 + \frac{3x_2}{2} + 1$

$$
gradient \ F = \{ 1, \frac{3}{2} \}
$$

Нарисуем этот вектор на плоскости. Кроме того, имеем линию уровня:

$$x_1 + \frac{3}{2}x_2 = c$$

Видим, что $gradient \ F$ перепендикулярен второй линии.

$$
F(2; \frac{2}{3}) = 2 + \frac{3}{2} * \frac{2}{3} + 1 = 4
$$

$$
F(0; 2) = 0 + \frac{3}{2} * 2 + 1 = 4
$$

Так как решением являются две точки, то \textbf{оптимальным вектором решений является их выпуклая линейная комбинация}:

$$
X^{*} = \alpha * A + (1 - \alpha) B, \ \ 0 \le \alpha \le 1
$$, где т. $A(2, \frac{2}{3})$, т. $B(0, 2)$

$$X^{*} = (2\alpha, \frac{2}{3}\alpha + 2(1 - \alpha))$$

$$
X_3^{*} = 2 - 2\alpha, X_{4}^{*} = 1 - \frac{1}{3} * 2\alpha - \frac{1}{2}(\frac{2}{3}\alpha + 2(1 - \alpha))
$$

\begin{figure}[ht]
    \centering
    \begin{tikzpicture}[scale=2]
        \draw[very thin,color=gray!30] (-0.5,-0.5) grid (3.5, 2.5);
        \draw[->] (-0.5,0) -- (3.5,0) node[right] {$x_1$};
        \draw[->] (0,-0.5) -- (0,2.5) node[above] {$x_2$};

        \coordinate[] (A) at (0, 2);
        \coordinate[] (B) at (2, 2/3);
        \coordinate[] (C) at (3, 0);
        \coordinate[] (D) at (2, 0);
        \coordinate[] (O) at (0, 0);

        \draw [thick, marked, extend=1cm] (C) node[below right, yshift=-0.3cm]{$(1)$} -- (A);
        \draw [thick, marked, extend=1cm] (D) node[below right, yshift=-0.5cm]{$(2)$} -- (B);
        \draw [marked] (O) -- (D);
        \draw [marked] (A) -- (O);

        \draw [very thick, color=red, pattern = north west lines, pattern color = gray] (O) -- (A) -- (B) -- (D) -- cycle;

        \draw [very thick, color=blue, extend] (A) -- (B);

        \draw[->, very thick, color=blue, markedP] (0,0) -- (2,3) node[above] {$\mathrm{grad} F$};

        \node[dot, color=red] at (A) {};
        \node[dot, color=red] at (B) {};

        \draw[dashed] (B) -- (0, 2/3) node[left]{$2/3$};

        \foreach \point in {(O), (C), (D)}
            \node[dot] at \point {};
        \foreach \x in {0,...,3}
            \draw (\x, 1pt) -- (\x, -1pt) node[below left] {$\x$};
        \foreach \y in {1,2}
            \draw (1pt, \y) -- (-1pt, \y) node[below left] {$\y$};
    \end{tikzpicture}
    \caption{Чертёж к примеру 3}
\end{figure}

\end{document}