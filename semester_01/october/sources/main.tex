\documentclass{article}
\usepackage[utf8]{inputenc}

\usepackage[T2A]{fontenc}
\usepackage[utf8]{inputenc}
\usepackage[russian]{babel}

\usepackage{tabularx}
\usepackage{amsmath}
\usepackage{pgfplots}
\usepackage{geometry}
\geometry{
    left=1cm,right=1cm,top=2cm,bottom=2cm
}
\newcommand*\diff{\mathop{}\!\mathrm{d}}

\newtheorem{definition}{Определение}
\newtheorem{theorem}{Теорема}

\DeclareMathOperator{\sign}{sign}

\usepackage{hyperref}
\hypersetup{
    colorlinks, citecolor=black, filecolor=black, linkcolor=black, urlcolor=black
}

\title{Теория принятия решений}
\author{Лисид Лаконский}
\date{October 2023}

\begin{document}
\raggedright

\maketitle

\tableofcontents
\pagebreak

\section{Лекция — 23.10.2023}

Докажем следующее утверждение: множество всех допустимых решений системы ограничений задачи линейного программирования является выпуклым множеством.

$F(\overline{x}) = \sum\limits_{i = 1}^{n} C_{i} x_{i} \to max (min)$

$AX = B$, $A = (a_{i j})_{m \times n}$, $X = (x_1, \dots, x_{n})^{T}$, $B = (b_1, \dots, b_{m})^{T}$

Пусть $X_{1}^{*}$, $X_{2}^{*}$ — решение системы ограничений, то есть, $A X_{1}^{*} = A X_{2}^{*} = B$

Покажем, что $X^{*} = \alpha X_{1}^{*} + (1 - \alpha) X_{2}^{*}$, где $0 \le \alpha \le 1$ также является решением системы ограничений.

$AX^{*} = A(\alpha X_{1}^{*} + (1 - \alpha) X_{2}^{*}) = \alpha * AX_{1}^{*} + (1 - \alpha) * AX_{2}^{*} = \alpha B + (1 - \alpha) B - B$

Так как $X^{*}$ — это выпуклое множество...

\subsection{Решение задач линейного программирования}

\begin{theorem}
Если задача линейного программирования (далее — ЗЛП) имеет оптимальное решение, то линейная функция $F(\overline{x})$ \textbf{достигает своего оптимума (то есть, максимума или минимума) в одной из угловых точек многогранника (многоугольника на плоскости) решений}.

Отметим, что \textbf{если оптимальное решение достигается более чем в одной точке}, то \textbf{оно является выпуклой линейной комбинацией этих точек}.
\end{theorem}

\begin{theorem}
    Каждому допустимому базисному решению $X^{*} = (x_1^*, x_2^*, x_n^*)$ ЗЛП \textbf{соответствует одна из угловых точек многогранника (многоугольника на плоскости) решений}.

    Таким образом, \textbf{любое оптимальное решение ЗЛП является одной из угловых точек}, то есть \textbf{допустимым базисным решением}.
\end{theorem}

\subsubsection{Геометрический способ решения}

Рассмотрим решение на плоскости. Оно возможно в следующих случаях:

\begin{enumerate}
    \item Имеем две переменные: $n = 2$
    \item Количество неизвестных минус количество уравнений равно двум: $n - m = 2$
\end{enumerate}

Чтобы показать геометрическое решение, необходимо выбрать $m$ переменных из $x_1, \dots, x_{n}$ в качестве основных (главных, базисных), а остальные переменные назовем свободными.

Пусть $x_1$, $x_2$ — свободные переменные, а $x_3, \dots, x_{n}$ — базисные.

$$
a_{i 1} x_{1} + \dots + a_{i n} x_{n} = b_{i} \implies
$$

$$
\begin{cases}
x_3 = \beta_{3} + \alpha_{3 1} x_1 + \alpha_{3 2} x_{2} \\
\dots \\
x_n = \beta_{n} + \alpha_{n 1} x_1 + \alpha_{n 2} x_{2}
\end{cases}
$$

Так как $x_{i} \ge 0$, $\forall i = \overline{1, n} \implies \begin{cases}
    \beta_{i} + \alpha_{i 1} x_1 + \alpha_{i 2} x_2 \ge 0 \\
    x_1, x_2 \ge 0
\end{cases}$

\textbf{Нужно показать геометрически множество решений этой системы}, которое является выпуклым многоугольником.

\begin{enumerate}
    \item \textbf{Если нет пересечений в системе ограничений, то решение не существует}.
    \item \textbf{Если есть пересечение, то решение будет}.
\end{enumerate}

$F(\overline{x})$ выше было записано в следующем виде: $F(\overline{x}) = c_1 x_1 + c_2 x_2 + \dots + c_n x_n$. \textbf{Необходимо переписать ее, выразив базисные переменные через свободные в следующем виде}: $F(\overline{x}) = \gamma_1 x_1 + \gamma_2 x_2 + \gamma_{0}$

Если рассматривается задача на максимум, то \textbf{решение нужно искать в направлении возрастания функции} $F(\overline{x})$. Направление возрастания функции $F(\overline{x})$ — вектор градиент: $grad \ F = \{ \frac{\delta F}{\delta x_1}, \dots, \frac{\delta F}{\delta x_{n}} \}$. В нашем случае: $grad \ F = \{ \gamma_1, \gamma_2 \}$.

То есть, \textbf{необходимо нарисовать на плоскости $x_1 o x_2$ уравнение прямых} $\gamma_1 x_1 + \gamma_2 x_2 = c$ (опорные прямые), перпендикулярные $grad \ F$

\textbf{Если мы хотим найти максимум, то на многоугольнике решений надо найти точку, через которую проходит линия уровня (то есть, опорная прямая) с наибольшим значением уровня} (то есть, константы $c$). Получающаяся точка и является оптимальным решением.

В случае \textbf{задачи на минимум} необходимо двигаться в направлении $- grad \ F$.

\paragraph{Пример №1}

Решить геометрически задачу линейного программирования:

$$F(\overline{x}) = 2 x_1 + 3x_2 \to max$$

При ограничениях:

$$
\begin{cases}
  x_1 + 3x_2 \le 18 \implies \frac{x_1}{18} + \frac{x_2}{6} \le 1\\
  2x_1 + x_2 \le 16 \implies \frac{x_1}{8} + \frac{x_2}{16} \le 1\\
  x_2 \le 5\\
  x_1 \le 7\\
  x_1, x_2 \ge 0  
\end{cases}
$$

Нарисуем все эти ограничения на плоскости $x_1 o x_2$. В результате получаем выпуклый шестиугольник.

$gradient \ F = (2, 3)$

Опорная прмая: $2x_1 + 3x_2 = c$

Рассмотрим:

$$
\begin{cases}
    x_1 + 3x_2 = 18 \implies x_1 = 18 - 3x_2 \\
    2x_1 + x_2 = 16 \implies 36 - 6x_2 + x_2 = 16 \implies x_2 = 4 \implies x_1 = 6
\end{cases}
$$

Получаем решение: $X^{*} = (6, 4), F(X^{*}) = 24$

\paragraph{Пример №2}

Найти геометрически решение ЗЛП:

$$
F(\overline{x}) = 4x_1 - 3x_2 - x_4 + x_5 \to min
$$

При ограничениях:

$$
\begin{cases}
    -x_1 + 3x_2 + x_4 = 13 \\
    4x_1 + x_2 + x_5 = 26 \\
    -2x_1 + x_2 + x_3 = 1 \\
    x_1 - 3x_2 + x_6 = 0 \\
    x_{i} \ge 0, i = \overline{1, 6}    
\end{cases}
$$

Мы можем найти решение графически, так как $n = 6$, $m = 4$, $n - m = 2$. Из этой системы мы можем получить:

$$
\begin{cases}
  x_4 = 13 + x_{1} - 3x_{2} \ge 0 \\
  x_5 = 26 - 4x_1 - x_2 \ge 0 \\
  x_3 = 1 + 2x_1 - x_2 \ge 0 \\
  x_6 = -x_1 + 3x_2 \ge 0 \\
  x_1 x_2 \ge 0  
\end{cases} \implies
\begin{cases}
    x_1 - 3x_2 \ge -13 \\
    4x_1 + x_2 \le 26 \\
    x_2 \le 1 + 2x_1\\
    x_1 \le 3x_2 \\
    x_1 x_2 \ge 0
\end{cases}
$$

Нарисуем все эти ограничения на плоскости $x_1 o x_2$. Получаем многоугольник. Преобразуем $F(\overline{x})$, чтобы определить, какая из угловых точек является решением:

$$
F(\overline{x}) = 4x_1 - 3x_2 - (13 + x_1 - 3x_2) + (26 - 4x_1 - x_2) = -x_1 - x_2 + 13
$$

Таким образом:

$$
\gamma_1 = 1, \gamma_2 = 2, \gamma_3 = 13
$$

$$
gradient \ F = \{ -1, -1 \}, - gradient \ F = \{ 1, 1 \}
$$

Построим его на графике вместе с опорными линиями в его направлении. Видим: $A = \{ 5, 6 \}, F(A) = -5 - 6 + 13 = 2$. Проверим другие точки: $B = \{ 6, 2 \}, F(B) = -6 - 2 + 13 = 5$, $C = \{ 2, 5 \}, F(C) = -2 - 5 + 13 = 6$. Окончательно убеждаемся: $A$ — т. $min$

$x_4 = x_5 = 0$, $x_3 = 1 + 2 * 5 - 6 = 5$, $x_6 = -5 + 3 * 6 = 13 \implies X^{*} = (5, 6, 5, 0, 0, 13), F(X^{*}) = 2$

\paragraph{Пример №3}

Найти геометрически решение ЗЛП:

$$
F(\overline{x}) = x_1 + 3x_2 + 3x_4 - x_3 \to max
$$

При ограничениях:

$$
\begin{cases}
    x_1 - 3x_2 + 3x_3 - 6x_4 = 0 \\
    3x_2 - 2x_3 + 6x_4 = 2 \\
    x_{i} \ge 0, i = \overline{1, 4}
\end{cases}
$$

Сложим вместе первое и второе уравнение:

$$
x_1 + x_3 = 2 \implies x_3 = 2 - x_1 \ge 0
$$

Сложим первое уравнение дважды и второе уравнение трижды:

$$
2x_1 + 3x_2 + 6x_4 = 6 \implies x_4 = 1 - \frac{1}{3} x_1 - \frac{1}{2}x_2 \ge 0
$$

Получили:

$$
\begin{cases}
    2 - x_1 \ge 0 \\
    \frac{x_1}{3} + \frac{x_2}{2} \le 1 \\
    x_1, x_2 \ge 0
\end{cases}
$$

Нарисуем все эти ограничения на плоскости $x_1 o x_2$. Получаем четырехугольник. Преобразуем $F(\overline{x})$, чтобы определить, какая из угловых точек является решением:

$F(\overline{x}) = x_1 + 3x_2 + 3 (1 - \frac{1}{3} x_1 - \frac{1}{2} x_2) - (2 - x_1) = x_1 + 3x_2 + 3 - x_1 - \frac{3x_2}{2} - 2 + x_1 = x_1 + \frac{3x_2}{2} + 1$

$$
gradient \ F = \{ 1, \frac{3}{2} \}
$$

Нарисуем этот вектор на плоскости. Кроме того, имеем линию уровня:

$$x_1 + \frac{3}{2}x_2 = c$$

Видим, что $gradient \ F$ перепендикулярен второй линии.

$$
F(2; \frac{2}{3}) = 2 + \frac{3}{2} * \frac{2}{3} + 1 = 4
$$

$$
F(0; 2) = 0 + \frac{3}{2} * 2 + 1 = 4
$$

Так как решением являются две точки, то \textbf{оптимальным вектором решений является их выпуклая линейная комбинация}:

$$
X^{*} = \alpha * A + (1 - \alpha) B, \ \ 0 \le \alpha \le 1
$$, где т. $A(2, \frac{2}{3})$, т. $B(0, 2)$

$$X^{*} = (2\alpha, \frac{2}{3}\alpha + 2(1 - \alpha))$$

$$
X_3^{*} = 2 - 2\alpha, X_{4}^{*} = 1 - \frac{1}{3} * 2\alpha - \frac{1}{2}(\frac{2}{3}\alpha + 2(1 - \alpha))
$$

\section{Практическое занятие — 27.10.2023}

\subsection{Решение задач линейного программирования}

\subsubsection{Геометрический способ решения}

\paragraph{Пример №4}

Найти геометрически решение ЗЛП:

$$
F(\overline{x}) = x_1 + 4x_2 + x_3 - x_4 \to max
$$

При ограничениях:

$$
\begin{cases}
    2x_1 - x_3 + x_4 = 4 \\
    x_1 - 2x_2 - 2x_3 + x_4 = -1 \\
    x_i \ge 0, i = \overline{1, 4}
\end{cases}
$$

$n = 4, m = 2 \implies n - m = 2$ — следовательно, геометрическое решение возможно.

Выберем переменные $x_1$, $x_2$ в качестве свободных, выразим через них переменные $x_3$, $x_4$: вычтем из первого уравнения второе, получим  

$$x_1 + x_3 + 2x_2 = 5 \implies x_3 = 5 - x_1 - 2x_2 \ge 0$$.

Вычтем из второго уравнения два первых уравнения:

$$-3x_1 - 2x_2 - x_4 = -9 \implies x_4 = 9 - 3x_1 - 2x_2 \ge 0$$

Таким образом, мы получили систему неравенств на две переменные $x_1$, $x_2$:

$$
\begin{cases}
    5 - x_1 - 2x_2 \ge 0 \\ 
    9 - 3x_1 - 2x_2 \ge 0 \\
    x_1, x_2 \ge 0
\end{cases} \implies \begin{cases}
    x_1 + 2x_2 \le 5 \\
    3x_1 + 2x_2 \le 9 \\
    x_1, x_2 \ge 0
\end{cases}
$$

Построим на плоскости $x_1 o x_2$ область, отвечающую двум данным неравенствам. Получаем четырехугольник. Чтобы найти оптимальное решение, необходимо найти градиент. Перепишем $F(\overline{x})$:

$$F(\overline{x}) = x_1 + 4x_2 + 5 - x_1 - 2x_2 - (9 - 3x_1 - 2x_2) = -4 + 3x_1 + 4x_2$$

$$gradient \ F = \{ 3, 4 \}$$

Обозначим на графике в качестве точки и проведем из нуля вектор. Кроме того, необходимо найти линию уровня, имеющую наибольшее пересечение с угловой точкой. Линия уровня в нашем случае имеет уравнение:

$$3x_1 + 4x_2 = c$$

Решение является т. $A$ — пересечение двух прямых:

$$
\begin{cases}
    x_1 + 2x_2 = 5 \\
    3x_1 + 2x_2 = 9
\end{cases} \implies \begin{cases}
    x_1 = 5 - 2x_2 \\
    4x_2 = 6
\end{cases} \implies \begin{cases}
    x_1 = 2 \\ 
    x_2 = \frac{3}{2}
\end{cases}
$$

$x_3, x_4$ в данной точке будет равняться нулю. Так что имеем вектор решения: $x^{*} = (2; \frac{3}{2}; 0; 0)$, $F(x^{*}) = -4 + 3 * 2 + 1 * \frac{3}{2} = 8$ — \textbf{максимальное значение}

\subsubsection{Аналитические методы решение}

Одним из аналитических методов решения ЗЛП является так называемый \textbf{симплекс-метод}. Его суть заключается в том, что мы обходим угловые точки, но делаем это не геометрически, а аналитическим способом. Для его реализации необходимо установить следующие элементы:

\begin{enumerate}
    \item \textbf{Способ определения} какого-либо изначального допустимого базисного решения — то есть, удовлетворяющего системе ограничений:
    
    $AX = B$

    $X = (\beta_1, \dots, \beta_m, 0, \dots, 0)$, $\beta_{i} \ge 0$, $\forall i = \overline{1, m}$ — допустимое базисное решение;
    \item \textbf{Набор правил, определющих переход} к наилучшему по сравнению с предыдущим решению;
    \item \textbf{Критерий проверки} оптимальности найденного решения.
\end{enumerate}

На начальном этапе необходимо выбрать $m$ базисных переменных и выразить эти переменные через оставшиеся, свободные (количество которых равно $n - m$)

Пусть базисными являются переменные $x_1, x_2, \dots, x_{m}$:

$$
x_{i} = \alpha i_{m + 1} x_{m + 1} + \dots + \alpha i_{n} + \beta i, \ i = \overline{1. m}
$$

Начальное допустимое базисное решение:

$$
X^{(0)} = \{ \beta_1, \beta_2, \dots, \beta_m, 0, \dots, 0 \}
$$

где 

$$x_{m + 1} = \dots = x_{n} = 0, \beta_{i} \ge 0, \forall i = \overline{1, m}$$

В изначальное уравнение подставляем базисные переменные, выраженные через свободные:

$$
F(\overline{x}) = \sum\limits_{i = m + 1}^{n} \gamma_{i} x_i + \gamma_0 \to max
$$

\textbf{Критерий оптимальности}: если все коэффициенты $\gamma_{i}$ в выражении $F(\overline{x})$ через свободные переменные будет отрицательным, то данное решение будет оптимальным; если же существуют $\gamma_{k} > 0$, то решение не является оптимальным. И номер $k$ показывает, какую переменную необходимо перевести в базис. Но в базисе \textbf{не может быть} больше $n$ переменных. Следовательно, необходимо убрать одну из предыдущих базисных переменных. Это и есть \textbf{переход к наилучшему по сравнению с предыдущим решению}.

\paragraph{Пример №1}

Решить аналитически ЗЛП:

$$
F(\overline{x}) = 2x_1 + 3x_2 \to max
$$

При ограничениях:

$$
\begin{cases}
    x_1 + 3x_2 \le 18 \\
    2x_1 + x_2 < 16 \\
    x_2 \le 5 \\
    3x_1 \le 21 \\
    x_1 x_2 \ge 0  
\end{cases}
$$

Мы не можем запустить симплекс-метод для данной системы неравенств. Необходимо выполнить переход к канонической ЗЛП:

$$
\begin{cases}
  x_1 + 3x_2 + x_3 = 18 \\
  2x_1 + x_2 + x_4 = 16 \\
  x_2 + x_5 = 5 \\
  3x_1 + x_6 = 21  
\end{cases}
$$

Все данные переменные неотрицательны. Далее необходимо выбрать базисные переменные. Пусть ими будут $x_3, x_4, x_5, x_6$, так как они легко выражаются через $x_1, x_2$:

$$
\begin{cases}
    x_3 = 18 - x_1 - 3x_2 \ge 0 \\
    x_4 = 16 - 2x_1 - x_2 \ge 0 \\
    x_5 = 5 - x_2 \ge 0 \\
    x_6 = 21 - 3x_1 \ge 0
\end{cases}
$$

Необходимо проверить решение на оптимальность. Для этого в $F(\overline{x})$ необходимо подставить только свободные переменные — так уже есть. Видим, что коэффициенты в $F(\overline{x}) = 2x_1 + 3x_2$ положительны.

Если $x_1 = x_2 = 0$, то $x^{(0)} = (0, 0, 18, 16, 5, 21)$ — допустимое базисное решение. Не является оптимальным, так как $\gamma_1 > 0$ и $\gamma_i > 0$

В базис вводят переменную, у которой $\gamma_i$ максимально. В нашем случае $max \ \gamma_i = \gamma_2 = 3$. Следовательно, вводим $x_2$ в базис. Подставим в систему выше $x_1 = 0$:

$$
\begin{cases}
  x_2 \le 6 \\
  x_2 \le 16 \\
  x_2 \le 5 \\
  \text{нет ограничений}: x_2 \ge 0
\end{cases}
$$

Надо выбрать минимальное ограничение: $x_2 \le 5$. Следовательно, с строчки $x_5 = 5 - x_2 \ge 0$ необходимо начать. Следовательно, заменим $x_5$ в базисе на $x_2$ (уберем $x_5$, введем $x_2$)

$$
\begin{cases}
    x_5 = 5 - x_5 \ge 0 \\
    x_3 = 18 - x_1 - 3 (5 - x_5) = 3 - x_1 + 3x_5 \ge 0\\
    x_4 = 16 - 2x_1 - (5 - x_5) = 11 - 2x_1 + x_5 \ge 0\\
    x_6 = 21 - 3x_1 \ge 0
\end{cases}
$$

$x_1$, $x_2$ — свободные переменные. Следовательно, $x^{(1)} = (0,5,3,11,0,21)$, $F(x^{0}) = 2x_1 + 3 (5 - x_5) = 15 + 2x_1 - 3x_5$.

Решение не является оптимальным, так как $\gamma_1 > 0$ — следовательно, $x_1$ переводим в базис.

$x_5 = 0$:

$$
\begin{cases}
    5 \ge 0 \\
    x_1 \le 3 \\
    x_1 \le \frac{11}{2} \\
    x_1 \le \frac{21}{3}  
\end{cases}
$$

Меньшим является $x_{1} \le 3$, соответствующее строке $x_3 = 18 - x_1 - 3 (5 - x_5) = 3 - x_1 + 3x_5 \ge 0$ в предыдущей системе. Следовательно, необходимо удалить $x_3$. Перепишем данное уравнение. Необходимо $x_1$ выразить через $x_3, x_5$:

$$\begin{cases}
    x_1 = 3 - x_3 + 3x_5 \ge 0 \\
    x_2 = 5 - x_5 \ge 0 \\
    x_4 = 11 - 2 (3 - x_3 + 3x_5) + x_5 = 5 + 2x_3 - 5x_5 \ge 0\\
    x_6 = 21 - 3 (3 - x_3 + 3x_5) = 12 + 3x_3 - 9x_5 \ge 0
\end{cases}$$

$x_3 = x_5 = 0 \implies x^{(2)} = (3,5,0,5,0,12)$

$F(\overline{x}) = 15 + 2 * (3 - x_3 + 3x_5) - 3x_5 = 21 - 2x_3 + 3x_5$. В этом решении $F(x^{(2)}) = 21 > F(x^{(1)})$. Решение неоптимально, необходимо переводить $x_5$ в базис.

Подставим $x_3 = 0$:

$$
\begin{cases}
    3 + 3x_5 \ge 0 \implies x_5 \ge -1 \implies \text{ограничений нет}\\
    x_5 \le 5 \\
    x_5 \le 1 \\
    x_5 \le \frac{12}{9} \le \frac{4}{3}
\end{cases}
$$

Меньшим является $x_{5} \le 1$, соответствующее строке $x_4 = 11 - 2 (3 - x_3 + 3x_5) + x_5 = 5 + 2x_3 - 5x_5 \ge 0$ в предыдущей системе. Необходимо $x_5$ выразить через $x_4$, $x_3$:

$$
x_5 = 1 + \frac{2}{5} x_3 + \frac{1}{5} x_4
$$

Теперь это уравнение подставим в оставшиеся:

$$
\begin{cases}
  x_1 = 3 - x_3 + 3 (1 + \frac{2}{5} x_3 + \frac{1}{5} x_4) = 6 + \frac{1}{5} x_3 - \frac{3}{5} x_4 \\
  x_2 = 5 - (1 + \frac{2}{5} x_3 + \frac{1}{5} x_4) = 4 - \frac{2}{5}x_3 + \frac{1}{5} x_4 \\
  x_6 = 12 + 3x_3 - 9 (1 + \frac{2}{5} x_3 + \frac{1}{5} x_4) = 3 - \frac{3}{5} x_3 + \frac{9}{5} x_4 \ge 0
\end{cases}
$$

$x_3 = x_4 = 0 \implies x^{(3)} = (6, 4, 0. 0, 1, 3)$

$F(\overline{x}) = 21 - 2x_3 + 3 (1 + \frac{2}{5} x_3 + \frac{1}{5} x_4) = 24 - \frac{4}{5} x_3 - \frac{3}{5} x_4$. \textbf{Оптимальное решение достигнуто}: $x^{*} = x^{(3)} = (6, 4, 0, 0, 1, 3)$. Решение исходной задачи: $x^{*}_{\text{исх}} = (6, 4)$

\paragraph{Пример №2}

Найти аналитически решение ЗЛП:

$$
F(\overline{x}) = x_1 + 4x_2 + x_3 - x_4 \to max
$$

При ограничениях:

$$
\begin{cases}
    2x_1 - x_3 + x_4 = 4 \\
    x_1 - 2x_2 - 2x_3 + x_4 = -1 \\
    x_i \ge 0, i = \overline{1, 4}
\end{cases}
$$

Выберем переменные $x_1$, $x_2$ в качестве свободных, выразим через них переменные $x_3$, $x_4$: вычтем из первого уравнения второе, получим  

$$x_1 + x_3 + 2x_2 = 5 \implies x_3 = 5 - x_1 - 2x_2 \ge 0$$.

Вычтем из второго уравнения два первых уравнения:

$$-3x_1 - 2x_2 - x_4 = -9 \implies x_4 = 9 - 3x_1 - 2x_2 \ge 0$$

Таким образом, мы получили систему неравенств на две переменные $x_1$, $x_2$.

$x_1 = x_2 = 0 \implies x^{(0)} = (0,0,5,9)$. Перепишем $F(\overline{x})$, получим $F(\overline{x}) = -4 + 3x_1 + 4x_2$ — решение неоптимально, $F(x^{(0)}) = -4$. Переводим в базис переменную $x_2$, так как у нее наибольший коэффициент.

$x_1 = 0$:

$$
\begin{cases}
    x_3 = 5 - 2x_2 \ge 0 \implies x_3 \le \frac{5}{2} \\
    x_4 = 3 - 2x_2 \ge 0 \implies x_2 \le \frac{3}{2}
\end{cases}
$$

Минимальное из них $\frac{5}{2}$. Следовательно, необходимо избавляться от $x_3$.

$$
\begin{cases}
x_2 = \frac{5}{2} - \frac{1}{2} x_1 - \frac{1}{2} x_3 \ge 0 \\
x_4 = 9 - 3x_1 - 2 (\frac{5}{2} - \frac{1}{2} x_1 - \frac{1}{2} x_3) = 4 - 2x_1 + x_3 \ge 0
\end{cases}
$$

Перепишем $F(\overline{x}) = -4 + 3x_1 + 4 (\frac{5}{2} - \frac{1}{2} x_1 - \frac{1}{2} x_3) = 6 + x_1 - 2x_3$, $x^{(1)} = (0,\frac{5}{2},0,4), F(x^{(1)}) = 6$. Решение неоптимально, так как имеем положительный коэффициент. Переведём $x_1$ в базис.

$x_3 = 0$:

$$
\begin{cases}
  x_2 = \frac{5}{2} - \frac{1}{2} x_1 \ge 0 \implies x_1 \le 5 \\
  x_4 = 4 - 2x_1 \ge 0 \implies x_1 \le 2 \text{ — минимальное}  
\end{cases}
$$

Следовательно, второе уравнение необходимо переписать. Получим:

$$
\begin{cases}
  x_1 = 2 + \frac{1}{2} x_3 - \frac{1}{2} x_4 \ge 0 \\
  x_2 = \frac{5}{2} - \frac{1}{2} (2 + \frac{1}{2} x_3 - \frac{1}{2} x_4) - \frac{1}{2} x_3 = \frac{3}{2} - \frac{3}{4} x_3 + \frac{1}{4} x_4 \ge 0  
\end{cases}
$$

Перепишем $F(\overline{x}) = 6 + 2 + \frac{1}{2} x_3 - \frac{1}{2} x_4 - 2x_3 = 8 - \frac{3}{2} x_3 - \frac{1}{2} x_4$, $x^{(2)} = (2, \frac{3}{2}, 0, 0)$, $F(x^{(2)}) = 8$. Оптимальное решение, так как все с отрицательным коэффициентом.

\subsubsection{Домашнее задание}

Решить геометрически и аналитически ЗЛП:

$$
F(\overline{x}) = x_1 + 3x_2 + 3x_4 \to max
$$

При ограничениях:

$$
\begin{cases}
    x_1 - 3x_2 + 3x_3 - 6x_4 = 0 \\
    3x_2 - 2x_3 + 6x_4 = 2 \\
    x_{i} \ge 0, i = \overline{1, 4}
\end{cases}
$$

\section{Практическое занятие — 30.10.2023}

\subsection{Решение задач линейного программирования}

Как было показано выше, при поиске $max$ функции $f(x)$ принцип оптимальности в симплекс методе выглядел следующим образом: если в выражении $f(x)$ через свободные переменные все коэффициенты при этих переменных отрицательные, то выбранное допустимое базисное решение будет оптимальным. Если же ищется минимум функции $f(x)$, то в соответствии с принципом оптимальности все коэффициенты при этих переменных должны быть положительными, что и определяет решение.

\subsubsection{Реализация симплекс-метода при помощи симплекс-таблиц}

Рассмотрим реализацию симплекс-метода при помощи реализации симплекс-таблиц. Пусть рассматривается каноническая задача линейного программирования, в которой имеется $n$ переменных и $m$ ограничений—равенств, при этом $m \le n$.

$F(\overline{x}) = \sum\limits_{i = 1}^{n} c_i x_i \to max \ (min)$

$AX = B, X \ge 0$, $A = (a_{i j})_{m \times m}$, $X = (x_1, \dots, x_{n})^{T}$, $B = (b_1, \dots, b_{m})^{T}$

Ход решения:

Некоторые $m$ переменных выбираются в качестве базисных. Их мы всегда можем выразить через свободные в следующем виде:
    
$x_{i} = \beta_{i} - \alpha_{i, m + 1} x_{m + 1} - \dots - \alpha_{i, n} x_{n}$, $i = \overline{1, m}$, где $x_{m + 1, \dots, x_{n}}$ — свободные переменные

$x_{m + 1} = \dots = x_{n} = 0 \implies X^{(0)} = (\beta_1^{(0)}, \dots, \beta_{m}^{(0)}, 0, \dots, 0)$, $X^{(0)}$ является допустимым решением, если все $\beta{i} \ge 0$

Пусть решение допустимо, преобразуем функцию $F(\overline{x})$. То есть, подставим в нее базисные переменные, выраженные через свободные:

$F(\overline{x}) = \sum\limits_{i = 1}^{m} c_{i} x_{i} + \sum\limits_{i = m + 1}^{n} c_{i} x_{i} = \sum\limits_{i = 1}^{m} c_i (\beta_i^{(0)} - \alpha^{(0)}_{i, m + 1} x_{m + 1} - \dots - \alpha^{(0)}_{i, n} x_{n}) + \sum\limits_{i = m + 1}^{n} c_i x_i = \sum\limits_{i = 1}^{m} c_i \beta_i^{(0)} + \sum\limits_{j = m + 1}^{n} (c_j - \sum\limits_{i = 1}^{m} a_{i, j}^{(0)} c_{i}) x_j = \Delta_0^{(0)} \pm \sum\limits_{j = m + 1}^{n} \Delta_{j}^{(0)} x_{j}$, $\Delta_{0}^{(0)}$ — свободный коэффициент, $\Delta_{j}^{(0)}$ — обозначение для коэффициентов.

$\Delta_{0} = \sum\limits_{i = 1}^{m} c_{i} \beta_{i} = C_{\delta} * X_{\delta}$

Знак плюс в формуле выше берется для задачи на минимум, знак минус берется для задачи на максимум. Можем переписать: $\Delta_{j}^{(k)} = c_{j} - \sum\limits_{i = 1}^{m} \alpha_{i j}^{(k)} c_{i} = c_{j} - C_{\delta} * X_{j}$ — для задачи на минимум, $\Delta_{j}^{(k)} = \sum\limits_{i = 1}^{m} \alpha_{i j}^{(k)} c_{i} - c_{j}  = C_{\delta} * X_{j} - c_{j}$ — для задачи на максимум.

$C_{\delta}$ состоит из коэффициентов $c_{i}$, соответствующих базисным переменным. $X_{\delta} = (\beta_{1}^{0}, \dots, \beta_{m}^{(0)})$, нули в него не входят.

Все коэффициенты системы ограничений, а также $c_{i}$, $\Delta_0$ и $\Delta_{j}$ удобно свести в \textbf{симплекс-таблицу}, которая отражает ход решения, достигнутое решение и признак оптимальности на каждом шаге алгоритма.

На нулевом шаге симплекс-таблица будет иметь следующий вид:

\textbf{Перерисовать с фотографии}

$\Delta_1 = \dots = \Delta_{m} = 0$ (для базисных переменных). Последний столбец $\Theta$ опредлеляет базисную переменную, которую необходимо вывести из базиса. \textbf{Признак оптимальности достигнутого решения}: если на некотором шаге $s$ все $\Delta_{j}^{(s)} \ge 0$, $j = \overline{1, n}$, то полученное допустимое базисное решение $X^{(s)}$ будет оптимальным, то есть $X^{(s)} = X^{*}$ и $F(X^{*}) = F^{*} = \Delta_{0}^{(S)} = max \ (min) \ F(\overline{x})$

Пусть на некотором шаге $s$ некоторые из $\Delta_{j}^{(s)}$ отрицательны, то есть $\Delta_{j} < 0$. Тогда решение не является оптимальным и необходимо перейти к новому допустимому базисному решению. Далее из множества отрицательных $\Delta_{j}$ выберем то, которое по модулю оптимально. Например, это будет $\Delta_{l}^{(S)}$. Тогда переменную $x_{l}$ необходимо перевести в базис. Соответствующий столбец $X_{l}$ называется \textbf{разрешающим столбцом}. 

Далее по каждой строке симплекс-таблицы находим ограничение на новую базисную переменную $x_{l}$. Для этого делим поэлементно значения $X_{\delta}$ на значения $X_{l}$

\textbf{Правило определения} $\theta_{i}$ на шаге $s$:

\begin{enumerate}
    \item Если $\beta_{i}^{(s)}$ и $a_{i, l}^{(s)}$ имеют разные знаки, то $\theta_{i}^{(s)} = \infty$ считается равной бесконечности (нет ограничений);
    \item Если $\beta_{i}^{(s)} = 0$ и $a_{i, l}^{(s)} < 0$, то также $\theta_{i}^{(s)} = \infty$;
    \item Если $a_{i, l}^{(s)} = 0$, то $\theta_{i}^{(s)} = \infty$;
    \item Если $\beta_{i}^{(s)} = 0$ и $a_{i, l}^{(s)} > 0$, то $\theta_{i}^{(s)} = 0$;
    \item Если $\beta_{i}^{(s)} $ и $a_{i, l}^{(s)}$ имеют одинаковые знаки, то $\theta_{i}^{(s)} = \frac{\beta_{i}^{(s)}}{a_{i, l}^{(s)}}$
\end{enumerate}

Далее из значений $\theta_{i}$ нужно выбрать минимальное, которое обозначим как $\theta_{k}$. $k$ обозначает номер базисной переменной, которую нужно вывести из базиса. Соответствующую строку $x_{k}$ называют \textbf{разрешающей строкой}. $\alpha_{k, l}$ называют \textbf{разрешающим элементом}.

Преобразование симплекс-таблицы начинается с разрешающей строки и элемент, который стоит на пересечении этой строки и $X_{l}$, называют \textbf{разрешающим элементом}. Необходимо:

\begin{enumerate}
    \item все элементы разрешающей строки кроме $c_{k}$ поделить на разрешающий элемент;
    \item в столбцах, соответствующих базисным переменным, поставить 0 и 1, причем 1 — напротив своей базисной переменной в столбце б., 0 — в остальных случаях;
    \item остальные элементы $\alpha_{i, j}$ новой симплекс-таблицы вычисляются следующим образом:
    
    $$\alpha_{i, j}^{(s + 1)} = \alpha_{i, j}^{(s)} - \frac{\alpha_{i, j}^{(s)} * \alpha_{k, j}^{(s)}}{\alpha_{k, l}^{(s)}}, \beta_{i}^{(s + 1)} = \beta_{i}^{(s)} - \frac{\alpha_{i, l}^{(s)} * \beta_{k}^{(s)}}{\alpha{k, l}^{(s)}}$$
    \item Для новой симплекс-таблицы вычисляются коэффициенты $\Delta_{0}^{(s + 1)}$ и $\Delta_{j}^{(s + 1)}$ (только для свободных переменных, так как для базисных равны нулю) и проверяем таблицу на оптимальность. То есть, если окажется, что все коэффициенты неотрицательны, то решение достигнуто. Иначе решение неоптимально и мы заново вычисляем ограничения, находим все необходимое и переходим к новой таблице.
\end{enumerate}

\subsubsection{Решение задач}

\paragraph{Пример №1} Решить ЗЛП при помощи симплекс-таблиц:

$$F(\overline{x}) = 2x_1 + 3x_2 \to max$$

При ограничениях:

$$
\begin{cases}
    x_1 + 3x_2 \le 18 \\
    2x_1 + x_2 \le 16 \\
    x_2 \le 5 \\
    3x_1 \le 21 \\
    x_{i} \ge 0, i = \overline{1, 2}
\end{cases}
$$

Перейдем от неравенств к равенствам (добавим дополнительные переменные):

$$
\begin{cases}
    x_1 + 3x_2 + x_3 = 18 \\
    2x_1 + x_2 + x_4 = 16 \\
    x_2 + x_5 = 5 \\
    3x_1 + x_6 = 21 \\
    x_{i} \ge 0, i = \overline{1, 6}
\end{cases}
$$

Представим это всё в виде симплекс-таблицы (№1):

$$
\begin{pmatrix}
    \text{б} & C_{\text{б}} & X_{\text{б}} & X_1 & X_2 & X_3 & X_4 & X_5 & X_6 & \theta \\
    x_3 & 0 & 18 & 1 & 3 & 1 & 0 & 0 & 0 & \frac{18}{3} = 6\\
    x_4 & 0 & 16 & 2 & 1 & 0 & 1 & 0 & 0 & \frac{16}{1} = 16\\
    x_5 & 0 & 5 & 0 & 1 & 0 & 0 & 1 & 0 & \frac{5}{1} = 5\\
    x_6 & 0 & 21 & 3 & 0 & 0 & 0 & 0 & 1 \\
    & & 0 & -2 & -3 & 0 & 0 & 0 & 0
\end{pmatrix}
$$

$X^{(0)} = (0, 0, X_{\text{б}}) = (0, 0,0, 18, 16, 5, 7)$

$\Delta_{0} = C_{\delta} * X_{\delta}$

$\delta_{j} = C_{\delta} * X_{j} - c_{j}$

Решение неоптимально. Переводим $x_2$ в базис. Разрешающий столбец — $X_{2}$. $\theta_{5}$ — минимум. Разрешающий элемент — 1 на пересеченини $x_5$ и $X_{2}$. Следовательно, $x_5$ выводим из базиса. Необходимо составить новую симплекс-таблицу (№2):

$$
\begin{pmatrix}
    \text{б} & C_{\text{б}} & X_{\text{б}} & X_1 & X_2 & X_3 & X_4 & X_5 & X_6 & \theta \\
    x_3 & 0 & 3 & 1 & 0 & 1 & 0 & -3 & 0 & \frac{3}{1} = 3\\
    x_4 & 0 & 11 & 2 & 0 & 0 & 1 & -1 & 0 & \frac{11}{2} =2\\
    x_2 & 3 & 5 & 0 & 1 & 0 & 0 & 1 & 0 \\
    x_6 & 0 & 21 & 3 & 0 & 0 & 0 & 0 & 1 & \frac{21}{3}\\
    \Delta_{j} & & 15 & -2 & 0 & 0 & 0 & 3 & 0
\end{pmatrix}
$$

Видим, что есть один отрицательный коэффициент: $\Delta_{1}$. Решение неоптимально. Смотрим на столбец $X_{1}$. Они все положительны кроме одного нуля. Делим столбец. Выбираем минимум — это будет $\theta_{1}$. Следовательно, разрешающий элемент — на пересечении $x_3$ и $X_1$. Необходимо составить новую симплекс-таблицу (№3):

$$
\begin{pmatrix}
    \text{б} & C_{\text{б}} & X_{\text{б}} & X_1 & X_2 & X_3 & X_4 & X_5 & X_6 & \theta \\
    x_1 & 2 & 3 & 1 & 0 & 1 & 0 & -3 & 0 \\
    x_4 & 0 & 5 & 0 & 0 & -2 & 1 & 5 & 0 \\ 
    x_2 & 3 & 5 & 0 & 1 & 0 & 0 & 1 & 0 \\
    x_6 & 0 & 12 & 0 & 0 & -3 & 0 & 9 & 1 \\
    \Delta_{j} & 21 & 0 & 0 & 2 & 0 & -3 & 0
\end{pmatrix}
$$

\end{document}