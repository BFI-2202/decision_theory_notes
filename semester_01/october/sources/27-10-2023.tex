\documentclass{article}
\usepackage[utf8]{inputenc}

\usepackage[T2A]{fontenc}
\usepackage[utf8]{inputenc}
\usepackage[russian]{babel}

\usepackage{tabularx}
\usepackage{amsmath}
\usepackage{pgfplots}
\usepackage{geometry}
\geometry{
    left=1cm,right=1cm,top=2cm,bottom=2cm
}
\newcommand*\diff{\mathop{}\!\mathrm{d}}

\newtheorem{definition}{Определение}
\newtheorem{theorem}{Теорема}

\DeclareMathOperator{\sign}{sign}

\usepackage{hyperref}
\hypersetup{
    colorlinks, citecolor=black, filecolor=black, linkcolor=black, urlcolor=black
}

\tikzset{
  dot/.style={circle, fill, inner sep=1.5pt},
  marked/.style={
    postaction={decorate},
    decoration={
        markings,
        mark=
        between positions 0 and 1 step 3mm
        with
        {
            \draw[thick, color=gray!70, shorten >= 0, shorten <= 0, -] (0,0) -- (0.1,0.2);
        }
    }
  },
  markedP/.style={
    postaction={decorate},
    decoration={
        markings,
        mark=
        between positions 0.05 and 0.95 step 8mm
        with
        {
            \draw[thick, shorten >= 0, shorten <= 0, -] (0,-0.5) -- (0,0.5);
        }
    }
  },
  extend/.style={shorten >= -#1, shorten <= -#1},
  extend/.default=0.7cm
}
\usetikzlibrary{decorations.markings, patterns, calc}

\title{Теория принятия решений}
\author{Лисид Лаконский}
\date{October 2023}

\begin{document}
\raggedright

\maketitle

\tableofcontents
\pagebreak

\section{Практическое занятие — 27.10.2023}

\subsection{Решение задач линейного программирования}

\subsubsection{Геометрический способ решения}

\paragraph{Пример №4}

Найти геометрически решение ЗЛП:

$$
F(\overline{x}) = x_1 + 4x_2 + x_3 - x_4 \to max
$$

При ограничениях:

$$
\begin{cases}
    2x_1 - x_3 + x_4 = 4 \\
    x_1 - 2x_2 - 2x_3 + x_4 = -1 \\
    x_i \ge 0, i = \overline{1, 4}
\end{cases}
$$

\begin{figure}[ht]
    \centering
    \begin{tikzpicture}[scale=2]
        \draw[very thin,color=gray!30] (-0.5,-0.5) grid (5.5, 5.5);
        \draw[->] (-0.5,0) -- (5.5,0) node[right] {$x_1$};
        \draw[->] (0,-0.5) -- (0,5.5) node[above] {$x_2$};

        \coordinate[] (A) at (2, 3/2);
        \coordinate[] (B) at (1, 2);
        \coordinate[] (C) at (3, 1);
        \coordinate[] (D) at (1, 3);
        \coordinate[] (E) at (3, 0);
        \coordinate[] (O) at (0, 0);

        \draw [thick, marked, extend=3cm] (C) -- (B);
        \draw [thick, marked, extend=1cm] (E) -- (D);

        \draw [very thick, color=red, pattern = north west lines, pattern color = gray] (O) -- (0, 5/2) -- (A) -- (E) -- cycle;

        \draw[->, very thick, color=blue, markedP] (0,0) -- (3, 4) node[above] {$\mathrm{grad} F$};

        \node[dot, color=red, label=above right:{$A$}] at (A) {};

        \draw[dashed] (B) -- (0, 2);
        \draw[dashed] (C) -- (0, 1);
        \draw[dashed] (C) -- (3, 0);
        \draw[dashed] (D) -- (0, 3);
        \draw[dashed] (D) -- (1, 0);

        \draw[marked] (B) -- (0, 5/2) -- (O) -- (E);

        \foreach \point in {(B), (C), (D), (E)}
            \node[dot] at \point {};
        \foreach \x in {0,...,5}
            \draw (\x, 1pt) -- (\x, -1pt) node[below left] {$\x$};
        \foreach \y in {1,...,5}
            \draw (1pt, \y) -- (-1pt, \y) node[below left] {$\y$};
    \end{tikzpicture}
    \caption{Чертёж к примеру 4}
\end{figure}

$n = 4, m = 2 \implies n - m = 2$ — следовательно, геометрическое решение возможно.

Выберем переменные $x_1$, $x_2$ в качестве свободных, выразим через них переменные $x_3$, $x_4$: вычтем из первого уравнения второе, получим  

$$x_1 + x_3 + 2x_2 = 5 \implies x_3 = 5 - x_1 - 2x_2 \ge 0$$.

Вычтем из второго уравнения два первых уравнения:

$$-3x_1 - 2x_2 - x_4 = -9 \implies x_4 = 9 - 3x_1 - 2x_2 \ge 0$$

Таким образом, мы получили систему неравенств на две переменные $x_1$, $x_2$:

$$
\begin{cases}
    5 - x_1 - 2x_2 \ge 0 \\ 
    9 - 3x_1 - 2x_2 \ge 0 \\
    x_1, x_2 \ge 0
\end{cases} \implies \begin{cases}
    x_1 + 2x_2 \le 5 \\
    3x_1 + 2x_2 \le 9 \\
    x_1, x_2 \ge 0
\end{cases}
$$

Построим на плоскости $x_1 o x_2$ область, отвечающую двум данным неравенствам. Получаем четырехугольник. Чтобы найти оптимальное решение, необходимо найти градиент. Перепишем $F(\overline{x})$:

$$F(\overline{x}) = x_1 + 4x_2 + 5 - x_1 - 2x_2 - (9 - 3x_1 - 2x_2) = -4 + 3x_1 + 4x_2$$

$$gradient \ F = \{ 3, 4 \}$$

Обозначим на графике в качестве точки и проведем из нуля вектор. Кроме того, необходимо найти линию уровня, имеющую наибольшее пересечение с угловой точкой. Линия уровня в нашем случае имеет уравнение:

$$3x_1 + 4x_2 = c$$

Решение является т. $A$ — пересечение двух прямых:

$$
\begin{cases}
    x_1 + 2x_2 = 5 \\
    3x_1 + 2x_2 = 9
\end{cases} \implies \begin{cases}
    x_1 = 5 - 2x_2 \\
    4x_2 = 6
\end{cases} \implies \begin{cases}
    x_1 = 2 \\ 
    x_2 = \frac{3}{2}
\end{cases}
$$

$x_3, x_4$ в данной точке будет равняться нулю. Так что имеем вектор решения: $x^{*} = (2; \frac{3}{2}; 0; 0)$, $F(x^{*}) = -4 + 3 * 2 + 1 * \frac{3}{2} = 8$ — \textbf{максимальное значение}

\subsubsection{Аналитические методы решение}

Одним из аналитических методов решения ЗЛП является так называемый \textbf{симплекс-метод}. Его суть заключается в том, что мы обходим угловые точки, но делаем это не геометрически, а аналитическим способом. Для его реализации необходимо установить следующие элементы:

\begin{enumerate}
    \item \textbf{Способ определения} какого-либо изначального допустимого базисного решения — то есть, удовлетворяющего системе ограничений:
    
    $AX = B$

    $X = (\beta_1, \dots, \beta_m, 0, \dots, 0)$, $\beta_{i} \ge 0$, $\forall i = \overline{1, m}$ — допустимое базисное решение;
    \item \textbf{Набор правил, определющих переход} к наилучшему по сравнению с предыдущим решению;
    \item \textbf{Критерий проверки} оптимальности найденного решения.
\end{enumerate}

На начальном этапе необходимо выбрать $m$ базисных переменных и выразить эти переменные через оставшиеся, свободные (количество которых равно $n - m$)

Пусть базисными являются переменные $x_1, x_2, \dots, x_{m}$:

$$
x_{i} = \alpha i_{m + 1} x_{m + 1} + \dots + \alpha i_{n} + \beta i, \ i = \overline{1. m}
$$

Начальное допустимое базисное решение:

$$
X^{(0)} = \{ \beta_1, \beta_2, \dots, \beta_m, 0, \dots, 0 \}
$$

где 

$$x_{m + 1} = \dots = x_{n} = 0, \beta_{i} \ge 0, \forall i = \overline{1, m}$$

В изначальное уравнение подставляем базисные переменные, выраженные через свободные:

$$
F(\overline{x}) = \sum\limits_{i = m + 1}^{n} \gamma_{i} x_i + \gamma_0 \to max
$$

\textbf{Критерий оптимальности}: если все коэффициенты $\gamma_{i}$ в выражении $F(\overline{x})$ через свободные переменные будет отрицательным, то данное решение будет оптимальным; если же существуют $\gamma_{k} > 0$, то решение не является оптимальным. И номер $k$ показывает, какую переменную необходимо перевести в базис. Но в базисе \textbf{не может быть} больше $n$ переменных. Следовательно, необходимо убрать одну из предыдущих базисных переменных. Это и есть \textbf{переход к наилучшему по сравнению с предыдущим решению}.

\paragraph{Пример №1}

Решить аналитически ЗЛП:

$$
F(\overline{x}) = 2x_1 + 3x_2 \to max
$$

При ограничениях:

$$
\begin{cases}
    x_1 + 3x_2 \le 18 \\
    2x_1 + x_2 < 16 \\
    x_2 \le 5 \\
    3x_1 \le 21 \\
    x_1 x_2 \ge 0  
\end{cases}
$$

Мы не можем запустить симплекс-метод для данной системы неравенств. Необходимо выполнить переход к канонической ЗЛП:

$$
\begin{cases}
  x_1 + 3x_2 + x_3 = 18 \\
  2x_1 + x_2 + x_4 = 16 \\
  x_2 + x_5 = 5 \\
  3x_1 + x_6 = 21  
\end{cases}
$$

Все данные переменные неотрицательны. Далее необходимо выбрать базисные переменные. Пусть ими будут $x_3, x_4, x_5, x_6$, так как они легко выражаются через $x_1, x_2$:

$$
\begin{cases}
    x_3 = 18 - x_1 - 3x_2 \ge 0 \\
    x_4 = 16 - 2x_1 - x_2 \ge 0 \\
    x_5 = 5 - x_2 \ge 0 \\
    x_6 = 21 - 3x_1 \ge 0
\end{cases}
$$

Необходимо проверить решение на оптимальность. Для этого в $F(\overline{x})$ необходимо подставить только свободные переменные — так уже есть. Видим, что коэффициенты в $F(\overline{x}) = 2x_1 + 3x_2$ положительны.

Если $x_1 = x_2 = 0$, то $x^{(0)} = (0, 0, 18, 16, 5, 21)$ — допустимое базисное решение. Не является оптимальным, так как $\gamma_1 > 0$ и $\gamma_i > 0$

В базис вводят переменную, у которой $\gamma_i$ максимально. В нашем случае $max \ \gamma_i = \gamma_2 = 3$. Следовательно, вводим $x_2$ в базис. Подставим в систему выше $x_1 = 0$:

$$
\begin{cases}
  x_2 \le 6 \\
  x_2 \le 16 \\
  x_2 \le 5 \\
  \text{нет ограничений}: x_2 \ge 0
\end{cases}
$$

Надо выбрать минимальное ограничение: $x_2 \le 5$. Следовательно, с строчки $x_5 = 5 - x_2 \ge 0$ необходимо начать. Следовательно, заменим $x_5$ в базисе на $x_2$ (уберем $x_5$, введем $x_2$)

$$
\begin{cases}
    x_5 = 5 - x_5 \ge 0 \\
    x_3 = 18 - x_1 - 3 (5 - x_5) = 3 - x_1 + 3x_5 \ge 0\\
    x_4 = 16 - 2x_1 - (5 - x_5) = 11 - 2x_1 + x_5 \ge 0\\
    x_6 = 21 - 3x_1 \ge 0
\end{cases}
$$

$x_1$, $x_2$ — свободные переменные. Следовательно, $x^{(1)} = (0,5,3,11,0,21)$, $F(x^{0}) = 2x_1 + 3 (5 - x_5) = 15 + 2x_1 - 3x_5$.

Решение не является оптимальным, так как $\gamma_1 > 0$ — следовательно, $x_1$ переводим в базис.

$x_5 = 0$:

$$
\begin{cases}
    5 \ge 0 \\
    x_1 \le 3 \\
    x_1 \le \frac{11}{2} \\
    x_1 \le \frac{21}{3}  
\end{cases}
$$

Меньшим является $x_{1} \le 3$, соответствующее строке $x_3 = 18 - x_1 - 3 (5 - x_5) = 3 - x_1 + 3x_5 \ge 0$ в предыдущей системе. Следовательно, необходимо удалить $x_3$. Перепишем данное уравнение. Необходимо $x_1$ выразить через $x_3, x_5$:

$$\begin{cases}
    x_1 = 3 - x_3 + 3x_5 \ge 0 \\
    x_2 = 5 - x_5 \ge 0 \\
    x_4 = 11 - 2 (3 - x_3 + 3x_5) + x_5 = 5 + 2x_3 - 5x_5 \ge 0\\
    x_6 = 21 - 3 (3 - x_3 + 3x_5) = 12 + 3x_3 - 9x_5 \ge 0
\end{cases}$$

$x_3 = x_5 = 0 \implies x^{(2)} = (3,5,0,5,0,12)$

$F(\overline{x}) = 15 + 2 * (3 - x_3 + 3x_5) - 3x_5 = 21 - 2x_3 + 3x_5$. В этом решении $F(x^{(2)}) = 21 > F(x^{(1)})$. Решение неоптимально, необходимо переводить $x_5$ в базис.

Подставим $x_3 = 0$:

$$
\begin{cases}
    3 + 3x_5 \ge 0 \implies x_5 \ge -1 \implies \text{ограничений нет}\\
    x_5 \le 5 \\
    x_5 \le 1 \\
    x_5 \le \frac{12}{9} \le \frac{4}{3}
\end{cases}
$$

Меньшим является $x_{5} \le 1$, соответствующее строке $x_4 = 11 - 2 (3 - x_3 + 3x_5) + x_5 = 5 + 2x_3 - 5x_5 \ge 0$ в предыдущей системе. Необходимо $x_5$ выразить через $x_4$, $x_3$:

$$
x_5 = 1 + \frac{2}{5} x_3 + \frac{1}{5} x_4
$$

Теперь это уравнение подставим в оставшиеся:

$$
\begin{cases}
  x_1 = 3 - x_3 + 3 (1 + \frac{2}{5} x_3 + \frac{1}{5} x_4) = 6 + \frac{1}{5} x_3 - \frac{3}{5} x_4 \\
  x_2 = 5 - (1 + \frac{2}{5} x_3 + \frac{1}{5} x_4) = 4 - \frac{2}{5}x_3 + \frac{1}{5} x_4 \\
  x_6 = 12 + 3x_3 - 9 (1 + \frac{2}{5} x_3 + \frac{1}{5} x_4) = 3 - \frac{3}{5} x_3 + \frac{9}{5} x_4 \ge 0
\end{cases}
$$

$x_3 = x_4 = 0 \implies x^{(3)} = (6, 4, 0. 0, 1, 3)$

$F(\overline{x}) = 21 - 2x_3 + 3 (1 + \frac{2}{5} x_3 + \frac{1}{5} x_4) = 24 - \frac{4}{5} x_3 - \frac{3}{5} x_4$. \textbf{Оптимальное решение достигнуто}: $x^{*} = x^{(3)} = (6, 4, 0, 0, 1, 3)$. Решение исходной задачи: $x^{*}_{\text{исх}} = (6, 4)$

\paragraph{Пример №2}

Найти аналитически решение ЗЛП:

$$
F(\overline{x}) = x_1 + 4x_2 + x_3 - x_4 \to max
$$

При ограничениях:

$$
\begin{cases}
    2x_1 - x_3 + x_4 = 4 \\
    x_1 - 2x_2 - 2x_3 + x_4 = -1 \\
    x_i \ge 0, i = \overline{1, 4}
\end{cases}
$$

Выберем переменные $x_1$, $x_2$ в качестве свободных, выразим через них переменные $x_3$, $x_4$: вычтем из первого уравнения второе, получим  

$$x_1 + x_3 + 2x_2 = 5 \implies x_3 = 5 - x_1 - 2x_2 \ge 0$$.

Вычтем из второго уравнения два первых уравнения:

$$-3x_1 - 2x_2 - x_4 = -9 \implies x_4 = 9 - 3x_1 - 2x_2 \ge 0$$

Таким образом, мы получили систему неравенств на две переменные $x_1$, $x_2$.

$x_1 = x_2 = 0 \implies x^{(0)} = (0,0,5,9)$. Перепишем $F(\overline{x})$, получим $F(\overline{x}) = -4 + 3x_1 + 4x_2$ — решение неоптимально, $F(x^{(0)}) = -4$. Переводим в базис переменную $x_2$, так как у нее наибольший коэффициент.

$x_1 = 0$:

$$
\begin{cases}
    x_3 = 5 - 2x_2 \ge 0 \implies x_3 \le \frac{5}{2} \\
    x_4 = 3 - 2x_2 \ge 0 \implies x_2 \le \frac{3}{2}
\end{cases}
$$

Минимальное из них $\frac{5}{2}$. Следовательно, необходимо избавляться от $x_3$.

$$
\begin{cases}
x_2 = \frac{5}{2} - \frac{1}{2} x_1 - \frac{1}{2} x_3 \ge 0 \\
x_4 = 9 - 3x_1 - 2 (\frac{5}{2} - \frac{1}{2} x_1 - \frac{1}{2} x_3) = 4 - 2x_1 + x_3 \ge 0
\end{cases}
$$

Перепишем $F(\overline{x}) = -4 + 3x_1 + 4 (\frac{5}{2} - \frac{1}{2} x_1 - \frac{1}{2} x_3) = 6 + x_1 - 2x_3$, $x^{(1)} = (0,\frac{5}{2},0,4), F(x^{(1)}) = 6$. Решение неоптимально, так как имеем положительный коэффициент. Переведём $x_1$ в базис.

$x_3 = 0$:

$$
\begin{cases}
  x_2 = \frac{5}{2} - \frac{1}{2} x_1 \ge 0 \implies x_1 \le 5 \\
  x_4 = 4 - 2x_1 \ge 0 \implies x_1 \le 2 \text{ — минимальное}  
\end{cases}
$$

Следовательно, второе уравнение необходимо переписать. Получим:

$$
\begin{cases}
  x_1 = 2 + \frac{1}{2} x_3 - \frac{1}{2} x_4 \ge 0 \\
  x_2 = \frac{5}{2} - \frac{1}{2} (2 + \frac{1}{2} x_3 - \frac{1}{2} x_4) - \frac{1}{2} x_3 = \frac{3}{2} - \frac{3}{4} x_3 + \frac{1}{4} x_4 \ge 0  
\end{cases}
$$

Перепишем $F(\overline{x}) = 6 + 2 + \frac{1}{2} x_3 - \frac{1}{2} x_4 - 2x_3 = 8 - \frac{3}{2} x_3 - \frac{1}{2} x_4$, $x^{(2)} = (2, \frac{3}{2}, 0, 0)$, $F(x^{(2)}) = 8$. Оптимальное решение, так как все с отрицательным коэффициентом.

\subsubsection{Домашнее задание}

Решить геометрически и аналитически ЗЛП:

$$
F(\overline{x}) = x_1 + 3x_2 + 3x_4 \to max
$$

При ограничениях:

$$
\begin{cases}
    x_1 - 3x_2 + 3x_3 - 6x_4 = 0 \\
    3x_2 - 2x_3 + 6x_4 = 2 \\
    x_{i} \ge 0, i = \overline{1, 4}
\end{cases}
$$

\end{document}