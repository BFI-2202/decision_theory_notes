\documentclass{article}
\usepackage[utf8]{inputenc}

\usepackage[T2A]{fontenc}
\usepackage[utf8]{inputenc}
\usepackage[russian]{babel}

\usepackage{tabularx}
\usepackage{amsmath}
\usepackage{pgfplots}
\usepackage{geometry}
\geometry{
    left=1cm,right=1cm,top=2cm,bottom=2cm
}
\newcommand*\diff{\mathop{}\!\mathrm{d}}

\newtheorem{definition}{Определение}
\newtheorem{theorem}{Теорема}

\DeclareMathOperator{\sign}{sign}

\usepackage{hyperref}
\hypersetup{
    colorlinks, citecolor=black, filecolor=black, linkcolor=black, urlcolor=black
}

\title{Теория принятия решений}
\author{Лисид Лаконский}
\date{October 2023}

\begin{document}
\raggedright

\maketitle

\tableofcontents
\pagebreak

\section{Практическое занятие — 30.10.2023}

\subsection{Решение задач линейного программирования}

Как было показано выше, при поиске $max$ функции $f(x)$ принцип оптимальности в симплекс методе выглядел следующим образом: если в выражении $f(x)$ через свободные переменные все коэффициенты при этих переменных отрицательные, то выбранное допустимое базисное решение будет оптимальным. Если же ищется минимум функции $f(x)$, то в соответствии с принципом оптимальности все коэффициенты при этих переменных должны быть положительными, что и определяет решение.

\subsubsection{Реализация симплекс-метода при помощи симплекс-таблиц}

Рассмотрим реализацию симплекс-метода при помощи реализации симплекс-таблиц. Пусть рассматривается каноническая задача линейного программирования, в которой имеется $n$ переменных и $m$ ограничений—равенств, при этом $m \le n$.

$F(\overline{x}) = \sum\limits_{i = 1}^{n} c_i x_i \to max \ (min)$

$AX = B, X \ge 0$, $A = (a_{i j})_{m \times m}$, $X = (x_1, \dots, x_{n})^{T}$, $B = (b_1, \dots, b_{m})^{T}$

Ход решения:

Некоторые $m$ переменных выбираются в качестве базисных. Их мы всегда можем выразить через свободные в следующем виде:
    
$x_{i} = \beta_{i} - \alpha_{i, m + 1} x_{m + 1} - \dots - \alpha_{i, n} x_{n}$, $i = \overline{1, m}$, где $x_{m + 1, \dots, x_{n}}$ — свободные переменные

$x_{m + 1} = \dots = x_{n} = 0 \implies X^{(0)} = (\beta_1^{(0)}, \dots, \beta_{m}^{(0)}, 0, \dots, 0)$, $X^{(0)}$ является допустимым решением, если все $\beta{i} \ge 0$

Пусть решение допустимо, преобразуем функцию $F(\overline{x})$. То есть, подставим в нее базисные переменные, выраженные через свободные:

$F(\overline{x}) = \sum\limits_{i = 1}^{m} c_{i} x_{i} + \sum\limits_{i = m + 1}^{n} c_{i} x_{i} = \sum\limits_{i = 1}^{m} c_i (\beta_i^{(0)} - \alpha^{(0)}_{i, m + 1} x_{m + 1} - \dots - \alpha^{(0)}_{i, n} x_{n}) + \sum\limits_{i = m + 1}^{n} c_i x_i = \sum\limits_{i = 1}^{m} c_i \beta_i^{(0)} + \sum\limits_{j = m + 1}^{n} (c_j - \sum\limits_{i = 1}^{m} a_{i, j}^{(0)} c_{i}) x_j = \Delta_0^{(0)} \pm \sum\limits_{j = m + 1}^{n} \Delta_{j}^{(0)} x_{j}$, $\Delta_{0}^{(0)}$ — свободный коэффициент, $\Delta_{j}^{(0)}$ — обозначение для коэффициентов.

$\Delta_{0} = \sum\limits_{i = 1}^{m} c_{i} \beta_{i} = C_{\delta} * X_{\delta}$

Знак плюс в формуле выше берется для задачи на минимум, знак минус берется для задачи на максимум. Можем переписать: $\Delta_{j}^{(k)} = c_{j} - \sum\limits_{i = 1}^{m} \alpha_{i j}^{(k)} c_{i} = c_{j} - C_{\delta} * X_{j}$ — для задачи на минимум, $\Delta_{j}^{(k)} = \sum\limits_{i = 1}^{m} \alpha_{i j}^{(k)} c_{i} - c_{j}  = C_{\delta} * X_{j} - c_{j}$ — для задачи на максимум.

$C_{\delta}$ состоит из коэффициентов $c_{i}$, соответствующих базисным переменным. $X_{\delta} = (\beta_{1}^{0}, \dots, \beta_{m}^{(0)})$, нули в него не входят.

Все коэффициенты системы ограничений, а также $c_{i}$, $\Delta_0$ и $\Delta_{j}$ удобно свести в \textbf{симплекс-таблицу}, которая отражает ход решения, достигнутое решение и признак оптимальности на каждом шаге алгоритма.

На нулевом шаге симплекс-таблица будет иметь следующий вид:

\textbf{Перерисовать с фотографии}

$\Delta_1 = \dots = \Delta_{m} = 0$ (для базисных переменных). Последний столбец $\Theta$ опредлеляет базисную переменную, которую необходимо вывести из базиса. \textbf{Признак оптимальности достигнутого решения}: если на некотором шаге $s$ все $\Delta_{j}^{(s)} \ge 0$, $j = \overline{1, n}$, то полученное допустимое базисное решение $X^{(s)}$ будет оптимальным, то есть $X^{(s)} = X^{*}$ и $F(X^{*}) = F^{*} = \Delta_{0}^{(S)} = max \ (min) \ F(\overline{x})$

Пусть на некотором шаге $s$ некоторые из $\Delta_{j}^{(s)}$ отрицательны, то есть $\Delta_{j} < 0$. Тогда решение не является оптимальным и необходимо перейти к новому допустимому базисному решению. Далее из множества отрицательных $\Delta_{j}$ выберем то, которое по модулю оптимально. Например, это будет $\Delta_{l}^{(S)}$. Тогда переменную $x_{l}$ необходимо перевести в базис. Соответствующий столбец $X_{l}$ называется \textbf{разрешающим столбцом}. 

Далее по каждой строке симплекс-таблицы находим ограничение на новую базисную переменную $x_{l}$. Для этого делим поэлементно значения $X_{\delta}$ на значения $X_{l}$

\textbf{Правило определения} $\theta_{i}$ на шаге $s$:

\begin{enumerate}
    \item Если $\beta_{i}^{(s)}$ и $a_{i, l}^{(s)}$ имеют разные знаки, то $\theta_{i}^{(s)} = \infty$ считается равной бесконечности (нет ограничений);
    \item Если $\beta_{i}^{(s)} = 0$ и $a_{i, l}^{(s)} < 0$, то также $\theta_{i}^{(s)} = \infty$;
    \item Если $a_{i, l}^{(s)} = 0$, то $\theta_{i}^{(s)} = \infty$;
    \item Если $\beta_{i}^{(s)} = 0$ и $a_{i, l}^{(s)} > 0$, то $\theta_{i}^{(s)} = 0$;
    \item Если $\beta_{i}^{(s)} $ и $a_{i, l}^{(s)}$ имеют одинаковые знаки, то $\theta_{i}^{(s)} = \frac{\beta_{i}^{(s)}}{a_{i, l}^{(s)}}$
\end{enumerate}

Далее из значений $\theta_{i}$ нужно выбрать минимальное, которое обозначим как $\theta_{k}$. $k$ обозначает номер базисной переменной, которую нужно вывести из базиса. Соответствующую строку $x_{k}$ называют \textbf{разрешающей строкой}. $\alpha_{k, l}$ называют \textbf{разрешающим элементом}.

Преобразование симплекс-таблицы начинается с разрешающей строки и элемент, который стоит на пересечении этой строки и $X_{l}$, называют \textbf{разрешающим элементом}. Необходимо:

\begin{enumerate}
    \item все элементы разрешающей строки кроме $c_{k}$ поделить на разрешающий элемент;
    \item в столбцах, соответствующих базисным переменным, поставить 0 и 1, причем 1 — напротив своей базисной переменной в столбце б., 0 — в остальных случаях;
    \item остальные элементы $\alpha_{i, j}$ новой симплекс-таблицы вычисляются следующим образом:
    
    $$\alpha_{i, j}^{(s + 1)} = \alpha_{i, j}^{(s)} - \frac{\alpha_{i, j}^{(s)} * \alpha_{k, j}^{(s)}}{\alpha_{k, l}^{(s)}}, \beta_{i}^{(s + 1)} = \beta_{i}^{(s)} - \frac{\alpha_{i, l}^{(s)} * \beta_{k}^{(s)}}{\alpha{k, l}^{(s)}}$$
    \item Для новой симплекс-таблицы вычисляются коэффициенты $\Delta_{0}^{(s + 1)}$ и $\Delta_{j}^{(s + 1)}$ (только для свободных переменных, так как для базисных равны нулю) и проверяем таблицу на оптимальность. То есть, если окажется, что все коэффициенты неотрицательны, то решение достигнуто. Иначе решение неоптимально и мы заново вычисляем ограничения, находим все необходимое и переходим к новой таблице.
\end{enumerate}

\subsubsection{Решение задач}

\paragraph{Пример №1} Решить ЗЛП при помощи симплекс-таблиц:

$$F(\overline{x}) = 2x_1 + 3x_2 \to max$$

При ограничениях:

$$
\begin{cases}
    x_1 + 3x_2 \le 18 \\
    2x_1 + x_2 \le 16 \\
    x_2 \le 5 \\
    3x_1 \le 21 \\
    x_{i} \ge 0, i = \overline{1, 2}
\end{cases}
$$

Перейдем от неравенств к равенствам (добавим дополнительные переменные):

$$
\begin{cases}
    x_1 + 3x_2 + x_3 = 18 \\
    2x_1 + x_2 + x_4 = 16 \\
    x_2 + x_5 = 5 \\
    3x_1 + x_6 = 21 \\
    x_{i} \ge 0, i = \overline{1, 6}
\end{cases}
$$

Представим это всё в виде симплекс-таблицы (№1):

$$
\begin{pmatrix}
    \text{б} & C_{\text{б}} & X_{\text{б}} & X_1 & X_2 & X_3 & X_4 & X_5 & X_6 & \theta \\
    x_3 & 0 & 18 & 1 & 3 & 1 & 0 & 0 & 0 & \frac{18}{3} = 6\\
    x_4 & 0 & 16 & 2 & 1 & 0 & 1 & 0 & 0 & \frac{16}{1} = 16\\
    x_5 & 0 & 5 & 0 & 1 & 0 & 0 & 1 & 0 & \frac{5}{1} = 5\\
    x_6 & 0 & 21 & 3 & 0 & 0 & 0 & 0 & 1 \\
    & & 0 & -2 & -3 & 0 & 0 & 0 & 0
\end{pmatrix}
$$

$X^{(0)} = (0, 0, X_{\text{б}}) = (0, 0,0, 18, 16, 5, 7)$

$\Delta_{0} = C_{\delta} * X_{\delta}$

$\delta_{j} = C_{\delta} * X_{j} - c_{j}$

Решение неоптимально. Переводим $x_2$ в базис. Разрешающий столбец — $X_{2}$. $\theta_{5}$ — минимум. Разрешающий элемент — 1 на пересеченини $x_5$ и $X_{2}$. Следовательно, $x_5$ выводим из базиса. Необходимо составить новую симплекс-таблицу (№2):

$$
\begin{pmatrix}
    \text{б} & C_{\text{б}} & X_{\text{б}} & X_1 & X_2 & X_3 & X_4 & X_5 & X_6 & \theta \\
    x_3 & 0 & 3 & 1 & 0 & 1 & 0 & -3 & 0 & \frac{3}{1} = 3\\
    x_4 & 0 & 11 & 2 & 0 & 0 & 1 & -1 & 0 & \frac{11}{2} =2\\
    x_2 & 3 & 5 & 0 & 1 & 0 & 0 & 1 & 0 \\
    x_6 & 0 & 21 & 3 & 0 & 0 & 0 & 0 & 1 & \frac{21}{3}\\
    \Delta_{j} & & 15 & -2 & 0 & 0 & 0 & 3 & 0
\end{pmatrix}
$$

Видим, что есть один отрицательный коэффициент: $\Delta_{1}$. Решение неоптимально. Смотрим на столбец $X_{1}$. Они все положительны кроме одного нуля. Делим столбец. Выбираем минимум — это будет $\theta_{1}$. Следовательно, разрешающий элемент — на пересечении $x_3$ и $X_1$. Необходимо составить новую симплекс-таблицу (№3):

$$
\begin{pmatrix}
    \text{б} & C_{\text{б}} & X_{\text{б}} & X_1 & X_2 & X_3 & X_4 & X_5 & X_6 & \theta \\
    x_1 & 2 & 3 & 1 & 0 & 1 & 0 & -3 & 0 \\
    x_4 & 0 & 5 & 0 & 0 & -2 & 1 & 5 & 0 \\ 
    x_2 & 3 & 5 & 0 & 1 & 0 & 0 & 1 & 0 \\
    x_6 & 0 & 12 & 0 & 0 & -3 & 0 & 9 & 1 \\
    \Delta_{j} & 21 & 0 & 0 & 2 & 0 & -3 & 0
\end{pmatrix}
$$


\end{document}